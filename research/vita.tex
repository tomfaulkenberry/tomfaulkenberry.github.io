% Created 2024-01-11 Thu 14:57
% Intended LaTeX compiler: pdflatex
\documentclass[article,10pt]{article}
\usepackage[utf8]{inputenc}
\usepackage[T1]{fontenc}
\usepackage{graphicx}
\usepackage{longtable}
\usepackage{wrapfig}
\usepackage{rotating}
\usepackage[normalem]{ulem}
\usepackage{amsmath}
\usepackage{amssymb}
\usepackage{capt-of}
\usepackage{hyperref}
\usepackage[left=1in,right=1in,bottom=1in,top=1in]{geometry}
\usepackage{fancyhdr}
\pagestyle{fancyplain}
\lfoot{Last updated \today} \cfoot{} \rfoot{\thepage}
\date{\today}
\title{}
\hypersetup{
 pdfauthor={},
 pdftitle={},
 pdfkeywords={},
 pdfsubject={},
 pdfcreator={Emacs 29.1 (Org mode 9.6.6)}, 
 pdflang={English}}
\begin{document}


\section*{Thomas J. Faulkenberry, Ph.D.}
\label{sec:org7986e21}

Department of Psychological Sciences\\[0pt]
Tarleton State University\\[0pt]
Box T-0820, Stephenville, TX 76402\\[0pt]
Phone: +1 (254) 968-9816\\[0pt]
Email: faulkenberry@tarleton.edu\\[0pt]
Website: \url{http://tomfaulkenberry.github.io}

\subsection*{Education}
\label{sec:org1c469af}
\begin{itemize}
\item Ph.D., Psychology, Texas A\&M University – Commerce, 2010
\item M.S., Mathematics, Oklahoma State University - 2002
\item B.S. with highest honors, Mathematics, Southeastern Oklahoma State University, 2000
\end{itemize}

\subsection*{Academic positions}
\label{sec:org53a66b3}
\begin{itemize}
\item 2019-present: Associate Professor (with tenure), Department of Psychological Sciences, Tarleton State University
\item 2013-2019: Assistant Professor, Department of Psychological Sciences, Tarleton State University
\item 2012-2013: Assistant Professor, Department of Mathematics, Texas A\&M University – Commerce
\item 2010-2012: Visiting Assistant Professor, Department of Psychology and Special Education, Texas A\&M University – Commerce
\item 2005-2010: Lecturer, Department of Mathematics, Texas A\&M University - Commerce
\end{itemize}

\subsection*{Honors and awards}
\label{sec:orgb04a9d1}
\begin{itemize}
\item Faculty Excellence in Scholarship Award, Tarleton State University College of Education, 2023
\item Virtual MathPsych People's Choice Award, Society for Mathematical Psychology, 2023
\item Elizabeth A. Dahl Award for Excellence in Undergraduate Research, Journal of Psychological Inquiry, 2023
\item O. A. Grant Excellence in Teaching Award, Tarleton State University College of Education, 2018
\item Faculty Excellence in Scholarship Award, Tarleton State University College of Education, 2016
\item President's All-Purple Award, Tarleton State University, 2016
\item Fellow, The Psychonomic Society, 2014
\item Texas A\&M University System Teaching Excellence Award, 2010-2011
\item Best Graduate Student Presentation, Annual Research Symposium, TAMU-C, 2010
\item O.H. Hamilton Fellowship in Mathematics, Oklahoma State University, 2002
\item S. J. Scroggs Distinguished Graduate Fellowship, Oklahoma State University, 2001
\item Regional University Scholarship (4 years), Southeastern Oklahoma State University, 1996
\end{itemize}

\subsection*{Administrative activities}
\label{sec:orgaa8e34c}
\begin{itemize}
\item 2021-present: Head, Department of Psychological Sciences, Tarleton State University
\item 2019-2021: Assistant Head, Department of Psychological Sciences, Tarleton State University
\item 2011-2012: Director, Center for Undergraduate Research and Creative Activities, TAMU-C
\item 2010-2011: Director, Math \& Science Teacher Preparation Program (LeoTEACH), TAMU-C
\end{itemize}

\subsection*{Books}
\label{sec:orga7f4da2}
\begin{enumerate}
\item Faulkenberry, T. J. (2022). \emph{Psychological Statistics: The Basics}. New York: Routledge. \url{https://doi.org/10.4324/9781003181828}
\end{enumerate}

\subsection*{Papers}
\label{sec:orgc527b91}
\begin{enumerate}
\item Faulkenberry, T. J. (in press). A note on the normality assumption for modeling constraint in cognitive individual differences. To appear in \emph{Metodoloski Zvezki: Advances in Methodology and Statistics}, \url{https://arxiv.org/abs/2112.05503}
\item Zapata, B., \& Faulkenberry, T. J. (in press). A diffusion model decomposition of the unit-decade compatibility effect in two-digit number comparison. To appear in \emph{Proceedings of the 21st International Conference on Cognitive Modeling (ICCM 2023)}. \url{https://psyarxiv.com/f6qpy}
\item Colpitts, K., Dias, J., Faulkenberry, T. J., \& Harris Bozer, A. (in press). Investigation of the relationship between perceived mental workload and chronic pain. To appear in \emph{Psi Chi Journal of Psychological Research}.
\item Faulkenberry, T. J., \& Brennan, K. B. (2023). Computing analytic Bayes factors from summary statistics in repeated-measures designs. \emph{Biometrical Letters, 60} (1), 1-21. \url{https://doi.org/10.2478/bile-2023-0001}
\item Brennan, K., Rutledge, M., \& Faulkenberry, T. J. (2023). Arithmetic operation signs elicit spatial associations: A confirmatory Bayesian analysis. \emph{Journal of Psychological Inquiry, 27} (1), 5-13. \url{https://www.psychinquiry.org/wp-content/uploads/2023/05/vol27n1.pdf}
\item Faulkenberrry, T. J., \& Bowman, K. A. (2023). Bayesian modeling of the latent structure of individual differences in the numerical size-congruity effect. \emph{Journal of Cognitive Psychology, 35} (2), 217-232. \url{https://doi.org/10.1080/20445911.2022.2136186}
\item Vogel, S., Faulkenberry, T. J., \& Grabner, R. (2021). Quantitative and qualitative differences in the canonical and the reverse distance effect and their selective association with arithmetic and mathematical competencies. \emph{Frontiers in Education: Educational Psychology, 6}: 655747. \url{https://doi.org/10.3389/feduc.2021.655747}
\item Faulkenberry, T. J. (2021). The Pearson Bayes factor: An analytic formula for computing evidential value from minimal summary statistics. \emph{Biometrical Letters}, \emph{58} (1), 1-26. \url{https://doi.org/10.2478/bile-2021-0001}
\item Nejman, J. A. \& Faulkenberry, T. J. (2020). Implicit priming reveals decomposed processing in fraction comparison. \emph{Journal of Psychological Inquiry}, \emph{24} (2), 17-23. \url{https://www.psychinquiry.org/wp-content/uploads/2021/01/vol24-2v3.pdf}
\item Faulkenberry, T. J. (2020). Estimating Bayes factors from minimal summary statistics in repeated-measures analysis of variance designs. \emph{Metodoloski Zvezki: Advances in Methodology and Statistics}, \emph{17}, 1-17.  \url{https://arxiv.org/abs/1905.05569}
\item Faulkenberry, T. J., Ly, A., \& Wagenmakers, E. J. (2020). Bayesian statistics in numerical cognition: A tutorial using JASP. \emph{Journal of Numerical Cognition}, \emph{6}, 231-259. \url{https://doi.org/10.5964/jnc.v6i2.288}
\item Faulkenberry, T. J., Cruise, A., \& Shaki, S. (2020). Task instructions modulate unit-decade binding in two-digit number representation. \emph{Psychological Research}, \emph{84}, 424-439. \url{https://doi.org/}\href{https://dx.doi.org/10.1007/s00426-018-1057-9}{10.1007/s00426-018-1057-9}
\item Faulkenberry, T. J. (2019). Estimating evidential value from ANOVA summaries: A comment on Ly et al. (2018). \emph{Advances in Methods and Practices in Psychological Science}, \emph{2}, 406-409. \url{https://doi.org/}\href{https://doi.org/10.1177/2515245919872960}{10.1177/2515245919872960}
\item Faulkenberry, T. J. (2019). A tutorial on generalizing the default Bayesian t-test via posterior sampling and encompassing priors. \emph{Communications for Statistical Applications and Methods}, \emph{26}, 217-238. \url{https://doi.org/}\href{https://doi.org/10.29220/CSAM.2019.26.2.217}{10.29220/CSAM.2019.26.2.217}
\item Frampton, A. R., \& Faulkenberry, T. J. (2018). Mental arithmetic processes: Testing the independence of encoding and calculation. \emph{Journal of Psychological Inquiry}, \emph{22}, 30-35. \url{https://www.psychinquiry.org/wp-content/uploads/2019/03/Vol22-1.pdf}
\item Faulkenberry, T. J., Vick, A. D., \& Bowman, K. A. (2018). A shifted Wald decomposition of the numerical size-congruity effect: Support for a late interaction account. \emph{Polish Psychological Bulletin}, \emph{49}, 391-397. \url{https://doi.org/}\href{http://dx.doi.org/10.24425/119507}{10.24425/119507}
\item Faulkenberry, T. J., Witte, M., \& Hartmann, M. (2018). Tracking the continuous dynamics of numerical processing: A brief review and editorial. \emph{Journal of Numerical Cognition}, \emph{4} (2), 271-285. \url{https://doi.org/}\href{http://dx.doi.org/10.5964/jnc.v4i2.179}{10.5964/jnc.v4i2.179}
\item Faulkenberry, T. J. (2018). Computing Bayes factors to measure evidence from experiments: An extension of the BIC approximation. \emph{Biometrical Letters}, \emph{55} (1), 31-43. \url{https://doi.org/}\href{https://doi.org/10.2478/bile-2018-0003}{10.2478/bile-2018-0003}
\item Faulkenberry, T. J. (2018). A simple method for teaching Bayesian hypothesis testing in the brain and behavioral sciences. \emph{Journal of Undergraduate Neuroscience Education}, \emph{16}, A126-A130. \url{http://www.funjournal.org/wp-content/uploads/2018/01/june-16-126.pdf?x91298}
\item Faulkenberry, T. J. (2017). A single-boundary accumulator model of response times in an arithmetic verification task. \emph{Frontiers in Psychology}, \emph{8:1225}. \url{https://doi.org/}\href{http://dx.doi.org/10.3389/fpsyg.2017.01225}{10.3389/fpsyg.2017.01225/}
\item Faulkenberry, T. J., Cruise, A., \& Shaki, S. (2017). Reversing the manual digit bias in two-digit number comparison. \emph{Experimental Psychology}, \emph{64}, 191-204. \url{https://doi.org/}\href{http://dx.doi.org/10.1027/1618-3169/a000365}{10.1027/1618-3169/a000365}
\item Sobel, K. V., Puri, A. M., Faulkenberry, T. J., \& Dague, T. D. (2017). Visual search for conjunctions of physical and numerical size shows that they are processed independently. \emph{Journal of Experimental Psychology: Human Perception \& Performance}, \emph{43}, 444-453. \url{https://doi.org/}\href{http://dx.doi.org/10.1037/xhp0000323}{10.1037/xhp0000323}
\item Faulkenberry, T. J., \& Tummolini, L. (2016). Commentary: Is there any Influence of Variations in Context on Object-Affordance Effects in Schizophrenia? Perception of Property and Goals of Action). \emph{Frontiers in Psychology}, \emph{7:1915}. \url{https://doi.org/}\href{http://dx.doi.org/10.3389/fpsyg.2016.01915}{10.3389/fpsyg.2016.01915}
\item Faulkenberry, T. J. (2016). Testing a direct mapping versus competition account of response dynamics in number comparison. \emph{Journal of Cognitive Psychology}, \emph{28}, 825-842. \url{https://doi.org/}\href{http://dx.doi.org/10.1080/20445911.2016.1191504}{10.1080/20445911.2016.1191504}
\item Sobel, K. V., Puri, A. M., \& Faulkenberry, T. J. (2016). Bottom-up and top-down attentional contributions to the size-congruity effect. \emph{Attention, Perception, \& Psychophysics}, \emph{78}, 1324-1336. \url{https://doi.org/}\href{http://dx.doi.org/10.3758/s13414-016-1098-3}{10.3758/s13414-016-1098-3}
\item Faulkenberry, T. J., Cruise, A., Lavro, D., \& Shaki, S. (2016). Response trajectories capture the continuous dynamics of the size congruity effect. \emph{Acta Psychologica}, \emph{163}, 114-123. \url{https://doi.org/}\href{http://dx.doi.org/10.1016/j.actpsy.2015.11.010}{10.1016/j.actpsy.2015.11.010}
\item Faulkenberry, T. J., Montgomery, S. A., \& Tennes, S. N. (2015). Response trajectories reveal the temporal dynamics of fraction representations. \emph{Acta Psychologica}, \emph{159}, 100-107. \url{https://doi.org/}\href{http://dx.doi.org/10.1016/j.actpsy.2015.05.013}{10.1016/j.actpsy.2015.05.013}
\item Faulkenberry, T. J., \& Rey, A. R. (2014). Extending the reach of mousetracking in numerical cognition: A comment on Fischer and Hartmann (2014). \emph{Frontiers in Psychology}, \emph{5}:1436. \url{https://doi.org/}\href{http://dx.doi.org/10.3389/fpsyg.2014.01436}{10.3389/fpsyg.2014.01436}
\item Faulkenberry, T. J. (2014). Hand movements reflect competitive processing in numerical cognition. \emph{Canadian Journal of Experimental Psychology}, \emph{68}, 147-151. \url{https://doi.org/}\href{http://dx.doi.org/10.1037/cep0000021}{10.1037/cep0000021}
\item Faulkenberry, T. J., \& Geye, T. L. (2014). The cognitive origins of mathematics learning disability: A review. \emph{The Rehabilitation Professional}, \emph{22} (1), 9-16.
\item Faulkenberry, T. J., \& Faulkenberry, E. D. (2013). Teaching integer arithmetic without rules: An embodied approach. \emph{Oklahoma Journal of School Mathematics}, \emph{5} (2), 5-14.
\item Faulkenberry, T. J., (2013). The conceptual/procedural distinction belongs to strategies, not tasks: A comment on Gabriel et al. (2013). \emph{Frontiers in Psychology}, \emph{4}:820. \url{https://doi.org/}\href{http://dx.doi.org/10.3389/fpsyg.2013.00820}{10.3389/fpsyg.2013.00820}
\item Faulkenberry, T. J., \& Montgomery, S. A. (2013). The primacy of fraction components in adults’ numerical judgements. In Reeder, S. L. and Matney, G. T. (Eds.). \emph{Proceedings of the 40th Annual Meeting of the Research Council on Mathematics Learning} (pp. 155-162). Tulsa, OK: RCML
\item Faulkenberry, T. J. (2013). How the hand mirrors the mind: The embodiment of numerical cognition. In Reeder, S. L. and Matney, G. T. (Eds.). \emph{Proceedings of the 40th Annual Meeting of the Research Council on Mathematics Learning} (pp. 205-212). Tulsa, OK: RCML
\item Faulkenberry, E. D., \& Faulkenberry, T. J. (2012). Do you see what I see? An exploration of self-perception in the classroom. In S. L. Reeder (Ed.), \emph{Proceedings of the 39th Annual Meeting of the Research Council on Mathematics Learning} (pp. 121-126). Charlotte, NC: RCML.
\item Faulkenberry, T. J., \& Pierce, B. H. (2011). Mental representations in fraction comparison: Holistic versus component-based strategies. \emph{Experimental Psychology}, \emph{58}, 480-489. \url{https://doi.org/}\href{http://dx.doi.org/10.1027/1618-3169/a000116}{10.1027/1618-3169/a000116}
\item Faulkenberry, T. J. (2011). Individual differences in mental representations of fraction magnitude. In S. Reeder (Ed.) \emph{Proceedings of the 38th Annual Meeting of the Research Council on Mathematics Learning} (pp. 136-143). Cincinnati, OH: RCML.
\item Faulkenberry, E. D., \& Faulkenberry, T. J. (2010). Transforming the way we teach function transformations. \emph{Mathematics Teacher}, \emph{104}, 29-33.
\item Faulkenberry, T. J. (2010). The working memory demands of simple fraction strate- gies. In S. Reeder (Ed.) \emph{Proceedings of the 37th Annual Meeting of the Research Council on Mathematics Learning} (pp. 84-89). Conway, AR: RCML.
\item Faulkenberry, E. D. \& Faulkenberry, T. J. (2006). Constructivism in mathematics education: A historical and personal perspective. \emph{The Texas Science Teacher}, \emph{35}, 17- 22.
\end{enumerate}

\subsection*{Preprints}
\label{sec:orge352416}
\begin{enumerate}
\item Faulkenberry, T. J. (2023). Closed-form approximations of the two-sample Pearson Bayes factor. \emph{arXiv}, \url{https://arxiv.org/2310.11313}
\item Bowman, K. A., \& Faulkenberry, T. J. (2020). Modeling response times in the size-congruity effect: Early versus late interaction. \emph{PsyArXiv}, \url{https://psyarxiv.com/dns4t/}
\end{enumerate}

\subsection*{Contributions to Open Science}
\label{sec:org1d83a48}
\begin{enumerate}
\item In 2022, I built \emph{PsyStat}, a free online statistical calculator to accompany my book \emph{Psychological Statistics: The Basics} (Routledge). One important feature is that the calculator gives users the ability to compute Bayes factors directly from summary statistics in common experimental designs, which directly applies much of my theoretical work in Bayesian statistics since 2018. The calculator can be accessed at \url{https://tomfaulkenberry.shinyapps.io/psystat}, and its source code is available at \url{https://github.com/tomfaulkenberry/statShinyApps}.

\item Since 2021, I have contributed 15 entries to the online \emph{The Book of Statistical Proofs} (\url{https://statproofbook.github.io/}). Topics have included theorems about computing Bayes factors, as well as proofs of various statistical properties of common response time models, including the Wald and ex-Gaussian distributions.

\item In 2020, I co-authored the book \emph{Learning Statistics with JASP: A Tutorial for Psychology and Other Beginners} with Danielle Navarro and David Foxcroft. This book and its source files are freely downloadable from \url{https://learnstatswithjasp.com} and is published under a Creative Commons BY-SA license (CC BY-SA) version 4.0.
\end{enumerate}

\subsection*{Abstracts, columns, and book reviews}
\label{sec:org9cc2852}
\begin{enumerate}
\item Faulkenberry, T. J. (2023). A hierarchical Bayesian extension of the censored shifted Wald model for response times. \emph{Abstracts of the Psychonomic Society}, \emph{28}, 74-75.
\item Zapata, B. E., \& Faulkenberry, T. J. (2023). A diffusion model decomposition of the unit-decade compatibility effect in two-digit number comparison. \emph{Abstracts of the Psychonomic Society}, \emph{28}, 145.
\item Faulkenberry, T. J. (2023). A Mathematician's Apology: How a Life in Mathematics Has Shaped a Career in Psychology. \emph{Southwestern Psychologist}, \emph{16} (1). \url{https://rb.gy/7gpbb}
\item Faulkenberry, T. J., \& Scheuler, B. (2022). Testing the independence of encoding and calculation in mental addition: A confirmatory Bayesian analysis. \emph{Abstracts of the Psychonomic Society}, \emph{27}, 93.
\item Scheuler, B., \& Faulkenberry, T. J. (2022). Evaluating classical maximum likelihood estimation for estimating shifted-Wald models of response times. \emph{Abstracts of the Psychonomic Society}, \emph{27}, 181.
\item Faulkenberry, T. J. (2022). Message from the President. \emph{Southwestern Psychologist}, \emph{15} (2). \url{https://rb.gy/fn56p}
\item Faulkenberry, T. J. (2020). Getting started with Bayesian statistics. \emph{Southwestern Psychologist}, \emph{13} (3). \url{https://rb.gy/rikuim}
\item Bowman, K. A., \& Faulkenberry, T. J. (2020). Response time modeling for the size-congruity effect: Early vs. late interaction. \emph{Abstracts of the Psychonomic Society}, \emph{25}, 193.
\item Faulkenberry, T. J. (2020). Book review of "Chi-squared data analysis and model testing for beginners. \emph{MAA Reviews}, \url{https://www.maa.org/press/maa-reviews/chi-squared-data-analysis-and-model-testing-for-beginners}.
\item Faulkenberry, T. J. (2020). Statistics education awards presented at Joint Mathematics Meetings. \emph{MAA Focus}, \emph{40(2)}, 40. \url{https://www.maa.org/press/periodicals/maa-focus}
\item Faulkenberry, T. J. (2020). Closed form Bayes factor techniques for measuring evidential value from analysis of variance models. \emph{Abstracts of Papers Presented to the American Mathematical Society.}, \emph{41}, 256.
\item Faulkenberry, T. J. (2019). Book review of "Handbook of Approximate Bayesian Computation". \emph{MAA Reviews}, \url{https://www.maa.org/press/maa-reviews/handbook-of-approximate-bayesian-computation}.
\item Bowman, K. A., \& Faulkenberry, T. J. (2019). Response time modeling supports a late interaction account of the size-congruity effect. \emph{Abstracts of the Psychonomic Society}, \emph{24}, 227-228.
\item Faulkenberry, T. J. (2019). Treasurer's Column: Financial Challenges in Albuquerque. \emph{Southwestern Psychologist}, \emph{12(2)}, 3.
\item Faulkenberry, T. J. (2018). Modeling individual difference structures in the size-congruity effect. \emph{Abstracts of the Psychonomic Society}, \emph{23}, 42.
\item Bowman, K. A., \& Faulkenberry, T. J. (2018). Nonwords induce reverse priming effects in a lexical decision task. \emph{Abstracts of the Psychonomic Society}, \emph{23}, 246.
\item Faulkenberry, T. J. (2018). Treasurer's Column: Where does the money go? A quick picture of SWPA finances. \emph{Southwestern Psychologist}, \emph{11(1)}, 3.
\item Faulkenberry, T. J. (2017). A single-boundary accumulator model of decisions in a mental arithmetic task. \emph{Abstracts of the Psychonomic Society}, \emph{22}, 27.
\item Geye, T. L., \& Faulkenberry, T. J. (2017). Computer mousetracking reveals the facilitation and interference components of the size congruity effect. \emph{Abstracts of the Psychonomic Society}, \emph{22}, 106.
\item Bowman, K. A., \& Faulkenberry, T. J. (2017). The dynamics of spatial-operational momentum in mental arithmetic. \emph{Abstracts of the Psychonomic Society}, \emph{22}, 188.
\item Faulkenberry, T. J. (2017). Treasurer's Column: Standing on the shoulders of giants. \emph{Southwestern Psychologist}, \emph{10(2)}, 5.
\item Faulkenberry, T. J. (2016). Motor dynamics support a competition model of number processing. \emph{Abstracts of the Psychonomic Society}, \emph{21}, 26.
\item Bowman, K. A., \& Faulkenberry, T. J. (2016). Testing competing models of two-digit number representation: Decomposed versus holistic processing. \emph{Abstracts of the Psychonomic Society}, \emph{21}, 285.
\item Faulkenberry, T. J. (2016). Decoding the development of mathematical thinking: A book review of \emph{Development of Mathematical Thinking: Neural Substrates and Genetic Influences}. \emph{PsycCRITIQUES}, \emph{61} (31). doi: \href{http://dx.doi.org/10.1037/a0040434}{10.1037/a0040434}
\item Faulkenberry, T. J. (2016). Undergraduate students: An endangered resource? \emph{Southwestern Psychologist}, \emph{9(1)}, 2.
\item Faulkenberry, T. J., Cruise, A., Lavro, D., \& Shaki, S. (2015). Response trajectories support a late-interaction model of the size-congruity effect. \emph{Canadian Journal of Experimental Psychology, 69}, 346.
\item Faulkenberry, T. J., Cruise, A., \& Shaki, S. (2015). Reversing the manual decade bias in two-digit number comparison. \emph{Abstracts of the Psychonomic Society, 20}, 39.
\item Geye, T. L, \& Faulkenberry, T. J. (2015). Response trajectories capture individual differences in a size congruity task. \emph{Abstracts of the Psychonomic Society, 20}, 249.
\item Faulkenberry, T. J., Cruise, A., Lavro, D., \& Shaki, S. (2014). Response trajectories capture the continuous dynamics of the size-congruity effect. \emph{Abstracts of the Psychonomic Society, 19}, 53.
\item Faulkenberry, T. J. (2013). Measuring the working memory requirements of mental arithmetic. \emph{Canadian Journal of Experimental Psychology, 67}, 281.
\item Faulkenberry, T. J. (2013). Measuring the working memory requirements of mental arithmetic. \emph{Abstracts of the Psychonomic Society, 18}, 203-204.
\item Faulkenberry, T. J. (2012). The temporal dynamics of fraction representations: Components are processed first. \emph{Canadian Journal of Experimental Psychology, 66}, 310.
\item Faulkenberry, T. J. \& Montgomery, S. A. (2012). The primacy of components in numerical fractions. \emph{Abstracts of the Psychonomic Society, 17}, 206.
\item Faulkenberry, T. J. (2011). Brain-based mathematics: Promising practice or hopeful hype? \emph{RCML Intersection Points, 35} (3), 9-10.
\item Faulkenberry, T. J. \& Kelsey, A. R. (2011). Working memory and strategic performance in fraction comparison. \emph{Canadian Journal of Experimental Psychology, 65}, 311-311.
\item Faulkenberry, T. J. (2011). The dynamics of the SNARC effect: Evidence from mouse tracking. \emph{Canadian Journal of Experimental Psychology, 65}, 316-316.
\item Faulkenberry, T. J. (2011). Motor dynamics in numerical representations: Evidence from mouse tracking. \emph{Abstracts of the Psychonomic Society, 16}, 76-76.
\item Faulkenberry, T. J. (2010). The roles of phonological and visuo-spatial working memory resources in simple fraction strategies. \emph{Canadian Journal of Experimental Psychology, 64}, 302-302.
\item Lu, S. Wakefield, L. \& Faulkenberry, T. J. (2006). The roles of beginnings, overlap, and ends in event temporal relations. \emph{Abstracts of the Psychonomic Society, 11}, 9-9.
\end{enumerate}

\subsection*{Conference Presentations}
\label{sec:org54da339}
\begin{enumerate}
\item Faulkenberry, T. J. (April 2023). Attenuation of evidence in Bayesian repeated-measures analysis of variance. Southwestern Psychological Association. Frisco, TX.
\item Brennan, K., \& Faulkenberry, T. J. (April 2023). Operator priming effects in multiplication: Evidence of absence or absence of evidence? Southwestern Psychological Association. Frisco, TX.
\item Scheuler, B., Faulkenberry, T. J., \& Houpt, J. (April 2023). Evaluating single-level and hierarchical maximum likelihood estimation in shifted-Wald models. Southwestern Psychological Association. Frisco, TX.
\item Zapata, B., \& Faulkenberry, T. J. (April 2023). A diffusion model decomposition of the unit decade compatibility effect in two digit number comparison. Southwestern Psychological Association. Frisco, TX.
\item Faulkenberry, T. J. (March 2023). Gamma function approximations for computing closed-form Bayes factors. Mathematical Association of America Texas Section Meeting. Stephenville, TX.
\item Faulkenberry, T. J. (February 2023). Using computational mathematics as a research bridge between multiple disciplines. PERS Symposium, Tarleton State University, Stephenville, TX.
\item Faulkenberry, T. J. (October 2022). Bayesian hierarchical modeling of individual differences structures in numerical cognition. Southwest Cognition Conference (ARMADILLO), Tarleton State University, Stephenville, TX.
\item Scheuler, B., \& Faulkenberry, T. J. (October 2022). Classical maximum likelihood estimation in shifted-Wald models. Southwest Cognition Conference (ARMADILLO), Tarleton State University, Stephenville, TX.
\item Brennan, K., \& Faulkenberry, T. J. (October 2022). Operator preview effects in mental multiplication: Evidence for absence, or absence of evidence? Southwest Cognition Conference (ARMADILLO), Tarleton State University, Stephenville, TX.
\item Zapata, B., \& Faulkenberry, T. J. (October 2022). A diffusion model decomposition of the latent cognitive processes in two-digit number comparison. Southwest Cognition Conference (ARMADILLO), Tarleton State University, Stephenville, TX.
\item Codreanu, M., \& Faulkenberry, T. J. (April 2022). Using ex-Gaussian modeling to reveal mechanisms of the flanker effect, Southwestern Psychological Association, Baton Rouge, LA.
\item Zapata, B., Bowman, K., \& Faulkenberry, T. J. (April 2022). Response time modeling reveals the latent cognitive processes in two-digit number comparison, Southwestern Psychological Association, Baton Rouge, LA.
\item Scheuler, B., \& Faulkenberry, T. J. (April 2022). Cognitive processes in mental arithmetic: A confirmatory Bayesian analysis, Southwestern Psychological Association, Baton Rouge, LA.
\item Faulkenberry, T. J. (February 2022). Developing an interdisciplinary research experience for undergraduates in computational mathematics. PERS Symposium, Tarleton State University, Stephenville, TX.
\item Codreanu, M., \& Faulkenberry, T. J. (February 2022). Using ex-Gaussian and diffusion modeling to reveal mechanisms of the flanker effect, PERS Symposium, Tarleton State University, Stephenville, TX.
\item Scheuler, B., \& Faulkenberry, T. J. (February 2022). Cognitive processes in mental arithmetic: A confirmatory Bayesian analysis, PERS Symposium, Tarleton State University, Stephenville, TX.
\item Zapata, B., \& Faulkenberry, T. J. (February 2022). Response time modeling reveals the latent cognitive processes in two-digit number comparison, PERS Symposium, Tarleton State University, Stephenville, TX.
\item Faulkenberry, T. J. (September 2021). Obtaining closed form Bayes factors from summary statistics in common experimental designs. Applied Statistics 2021, Virtual/online.
\item Faulkenberry, T. J. (September, 2021). A Bayesian framework for modeling individual differences in two-digit number representation. Southwest Cognition Conference (ARMADILLO), Virtual/online.
\item Zapata, B., Bowman, K., \& Faulkenberry, T. J. (September 2021). An EZ Diffusion Model Parameter Decomposition of the Unit-decade Compatibility Effect. Southwest Cognition Conference (ARMADILLO), Virtual/online.
\item Jean Baptiste, C., Bowman, K., \& Faulkenberry, T. J. (September 2021). An ex-Gaussian decomposition of the unit-decade compatibility effect. Southwest Cognition Conference (ARMADILLO), Virtual/online.
\item Faulkenberry, T. J. \& Horry, R. (July, 2021). An interactive web applet for exploring the impact of unequal variances on the t-test. US Conference on the Teaching of Statistics (USCOTS 2021). Virtual/online.
\item Faulkenberry, T. J. (April, 2021). Some methods for approximating closed-form Bayes factors. Mathematical Association of America Texas Section Meeting. Virtual/online.
\item Faulkenberry, T. J. (April, 2021). Do asynchronous students perform worse? A Bayesian analysis of pandemic teaching. Southwestern Psychological Association. San Antonio, TX.
\item Faulkenberry, T. J., \& Bowman, K. A. (October 2020). Modeling a latent structure of individual differences in numerical cognition. Southwest Cognition Conference (ARMADILLO), Virtual/online.
\item Faulkenberry, T. J. (June 2020). A systems factorial technology approach to classifying the architecture of fraction perception. Math Cognition and Learning Society, Dublin, Ireland (cancelled due to COVID-19)
\item Scheuler, B., \& Faulkenberry, T. J. (April 2020). An illustration of Bayesian hypothesis testing: The case of the facial feedback effect. Southwestern Psychological Association, Frisco, TX (cancelled due to COVID-19)
\item Faulkenberry, T. J. (April 2020). Getting started with Bayesian inference in psychology: A workshop using JASP. Southwestern Psychological Association, Frisco, TX (cancelled due to COVID-19)
\item Bowman, K., Caldwell, K., Garcia, B., \& Faulkenberry, T. J. (April 2020). Maximum likelihood estimation of the Ex-Gaussian model for response time distributions. Southwestern Psychological Association, Frisco, TX (cancelled due to COVID-19)
\item Faulkenberry, T. J., (November, 2019). Org-mode and FoilTeX - an unlikely (but useful) combination for teaching", EmacsConf2019, Free Software Foundation, Virtual / Online.
\item Faulkenberry, T. J. (June, 2019). A hierarchical Bayesian model of individual difference structures for the size-congruity effect. Math Cognition and Learning Society, Ottawa, ON.
\item Faulkenberry, T. J., Hetzel, S., \& Bowman, K. (April, 2019). A systems factorial technology approach to classifying the architecture of fraction perception. Southwestern Psychological Association, Albuquerque, NM.
\item Faulkenberry, T. J. (April, 2019). An introduction to the theory and practice of Bayesian hypothesis testing: A workshop using JASP. Southwestern Psychological Association, Albuquerque, NM.
\item Bowman, K., \& Faulkenberry, T. J. (April, 2019). Response time modeling supports a late interaction account of the size-congruity effect. Southwestern Psychological Association, Albuquerque, NM.
\item Faulkenberry, T. J. (January, 2019). Demonstrating Bayesian model comparison with a class-sourced experiment in mental arithmetic. National Institute on the Teaching of Psychology (NITOP), St. Pete Beach, FL
\item Faulkenberry, T. J. (April, 2018). Introduction to Bayesian inference for the psychological sciences (workshop). Southwestern Psychological Association, Houston, TX
\item Bowman, K. A., \& Faulkenberry, T. J. (April, 2018). The dynamics of spatial operational momentum in mental arithmetic. Southwestern Psychological Association, Houston, TX.
\item Faulkenberry, T. J. (November, 2017). A hierarchical Bayesian model for measuring response times in a mental arithmetic task. Society for Mathematical Psychology, Vancouver, BC.
\item Faulkenberry, T. J. (April, 2017). Accumulator models of decision processes in mental arithmetic. Southwestern Psychological Association, San Antonio, TX
\item Faulkenberry, T. J., \& Wood, J. (April, 2017). A Bayesian perspective on the operator preview paradigm in mental arithmetic. Southwestern Psychological Association, San Antonio, TX
\item Nejman, J., \& Faulkenberry, T. J. (April, 2017). Implicit priming reveals both holistic and decomposed processing in fraction comparison. Southwestern Psychological Association, San Antonio, TX
\item Wood, J., \& Faulkenberry, T. J. (April, 2017). The dynamics of operator preview effects in mental arithmetic. Southwestern Psychological Association, San Antonio, TX
\item Bowman, K., \& Faulkenberry, T. J. (April, 2017). Testing competing models of two-digit number representation: Decomposed versus holistic processing. Southwestern Psychological Association, San Antonio, TX
\item Faulkenberry, T. J. (April, 2016). Testing two accounts of response dynamics in a number comparison task. Southwestern Psychological Association, Dallas, TX
\item Faulkenberry, T. J. (April, 2016). Recent developments on the size congruity effect in numerical cognition. Southwestern Psychological Association, Dallas, TX
\item Rutledge, M., \& Faulkenberry, T. J. (April, 2016). Spatial-numerical associations in mental arithmetic. Southwestern Psychological Association, Dallas, TX
\item Geye, T., \& Faulkenberry, T. J. (April, 2016). Computer mousetracking reveals individual differences in a size congruity task. Southwestern Psychological Association, Dallas, TX
\item Bowman, K. A., \& Faulkenberry, T. J. (April, 2016). The effects of mathematical fluency on multi-digit number representations. Southwestern Psychological Association, Dallas, TX
\item Faulkenberry, T. J. (October, 2015). Testing a direct-mapping versus competition account of response dynamics in a number comparison task. ARMADILLO 2015, Waco, TX.
\item Bowman, K. A., \& Faulkenberry, T. J. (October, 2015). The effects of mathematical fluency on multi-digit number representations. ARMADILLO 2015, Waco, TX.
\item Bowman, K. A., \& Faulkenberry, T. J. (October, 2015). The effects of mathematical fluency on multi-digit number representations. TAMUS Pathways Symposium, Corpus Christi, TX.
\item Bowman, K. A., \& Faulkenberry, T. J. (October, 2015). The effects of mathematical fluency on multi-digit number representations. Tarleton Research Symposium, Stephenville, TX.
\item Faulkenberry, T. J. (April, 2015). Class-sourcing replications of reaction time studies: An example in mathematical cognition. Southwestern Teachers of Psychology Conference, Wichita, KS.
\item Geye, T., Fleming, B., \& Faulkenberry, T. J. (April, 2015). Validation of the calculation fluency test for measuring arithmetic skills. Southwestern Psychological Association, Wichita, KS.
\item Frampton, A., \& Faulkenberry, T. J. (April, 2015). Cognitive arithmetic processs: The effects of problem size and format on performance. Southwestern Psychological Association, Wichita, KS.
\item Faulkenberry, T. J. (April, 2015). Evidence for a late-interactions model of the numerical size congruity effect. Southwestern Psychological Association, Wichita, KS.
\item Harris Bozer, A., \& Faulkenberry, T. J. (April, 2015). Applying the CREATE pedagogical tool to the online animal behavior course to enhance scientific literacy.  2015 CIRTL Forum: Preparing the Future STEM Faculty for the Rapidly Changing Landscape of Higher Education, College Station, TX.
\item Frampton, A., \& Faulkenberry, T. J. (March, 2015). Cognitive arithmetic processes: The effects of numerical surface form on strategy choice. Texas Undergraduate Research Day at the Capitol, Austin, TX.
\item Faulkenberry, E. D., Smith, K., Riggs, E., \& Faulkenberry, T. J. (February, 2015). The evolution of PST’s beliefs: Examining the effect of teacher preparation. Research Council on Mathematics Learning, Las Vegas, NV.
\item Faulkenberry, T. J. (October, 2014).  Hand movements reflect competitive processing in a numerical parity task. ARMADILLO 2014, Norman, OK.
\item Faulkenberry, T. J. (October, 2014). The dynamics of fraction representations: Components are processed first. ARMADILLO 2014, Norman, OK.
\item Faulkenberry, T. J. (April, 2014). Hand movements reflect competitive processing in numerical fraction representations. Southwestern Psychological Association, San Antonio, TX.
\item Faulkenberry, T. J. (April, 2014). A brief introduction to using R for teaching statistical methods. Southwestern Teachers of Psychology Conference, San Antonio, TX.
\item Faulkenberry, T. J. (March, 2014). A classroom activity for demonstrating confirmation bias. Tarleton Excellence in Teaching Conference, Stephenville, TX.
\item Smith, K. H., Riggs, B., Faulkenberry, E. D., \& Faulkenberry, T. J. (February, 2014). A snapshot of preservice teacher beliefs: A factor analytic method. Research Council on Mathematics Learning, San Antonio, TX.
\item Faulkenberry, T. J. (April, 2013). Modeling the roles of working memory and strategy type in fraction comparison. TX Section MAA Meeting, Texas Tech University, Lubbock, TX.
\item Faulkenberry, T. J. (March, 2013). Estimating the working memory requirements of mental arithmetic. OK-AR Section MAA Meeting, Oklahoma State University, Stillwater, OK.
\item Faulkenberry, T. J. (April, 2012). Some limitations in measuring working memory capacity. TX Section MAA Meeting, El Centro College, Dallas, TX.
\item Faulkenberry, T. J. (February, 2012). Examining the role of testing in learning mathematics: Directions for future research. 39th Annual Meeting of the Research Council on Mathematics Learning, Charlotte, NC.
\item Faulkenberry, T. J. \& Pierce, B. H. (October, 2011). The roles of working memory and strategy type in fraction comparison. ARMADILLO 2011, Commerce, TX.
\item Faulkenberry, T. J. (April, 2010). Working memory and strategy execution in simple fraction strategies. Annual Research Symposium, Texas A\& M University - Commerce.
\item Faulkenberry, T. J. (April, 2009). Mathematics anxiety among elementary education majors: Does test format matter?. Annual Research Symposium, Texas A\& M University - Commerce.
\item Faulkenberry, T. J. (February, 2009). Mathematics anxiety among elementary education majors. 36th Annual Meeting of the Research Council on Mathematics Learning, Rome, GA.
\item Faulkenberry, E. D. \& Faulkenberry, T. J. (February, 2008). An assessment of the mathematical knowledge of elementary preservice teachers with regard to number and operation. 35th Annual Meeting of the Research Council on Mathematics Learning, Oklahoma City, OK.
\item Faulkenberry, T. J. (February, 2008). Working memory: Cognitive and instructional implications for mathematics. 35th Annual Meeting of the Research Council on Mathematics Learning, Oklahoma City, OK.
\item Faulkenberry, E. D. \& Faulkenberry, T. J. (October, 2005). Using the geometry module in Teacher Quality grants. Charles A. Dana Center Higher Education Mathematics Conference, Austin, TX.
\item Faulkenberry, T. J. (April, 2005). Cognitive frameworks in advanced mathematical thinking. MAA Texas Section Meeting, University of Texas - Arlington.
\item Faulkenberry, T. J. (April, 2004). The shapes of 2-dimensional manifolds. MAA Texas Section Meeting, Texas A\&M University - Corpus Christi.
\item Faulkenberry, T. J. (March, 2003). Conway’s ZIP proof. MAA Oklahoma/Arkansas Section Meeting, University of Tulsa.
\item Faulkenberry, T. J. (March, 2002). Knot algorithms and their computational complexity. MAA Oklahoma/Arkansas Section Meeting, Henderson State University.
\item Faulkenberry, T. J. (March, 2002). Topology in the high school? National Council of Teachers of Mathematics Regional Conference, Oklahoma City, OK.
\item Faulkenberry, T. J. (March, 1999). The construction of a Riemann surface structure on a once-punctured torus. MAA Oklahoma/Arkansas Section Meeting, Arkansas Tech University.
\item Faulkenberry, T. J. (March, 1998). The classification of Markoff numbers on a once-punctured torus. MAA Oklahoma/Arkansas Section Meeting, Southern Nazarene University.
\end{enumerate}

\subsection*{Seminars and Invited Talks}
\label{sec:org9a8ee4e}
\begin{enumerate}
\item Faulkenberry, T. J. (April 2023). A Mathematician's Apology: How a Life in Mathematics has Shaped a Career in Psychology. Keynote Address - Southwestern Psychological Association, Frisco, TX.
\item Faulkenberry, T. J. (April 2023). Getting started with Bayesian Statistics: A workshop using JASP. Southwestern Psychological Association, Frisco, TX.
\item Faulkenberry, T. J. (August 2022). "Bayes"-ic Statistical Inference. CEE Professional Development Workshop, Tarleton State University, Stephenville, TX.
\item Faulkenberry, T. J. (April 2022). Workshop: Getting started in Bayesian Statistics with JASP. Southwestern Psychological Association, Baton Rouge, LA.
\item Faulkenberry, T. J. (September 2021). A Bayesian framework for modeling individual differences in numerical cognition. Educational Psychology Colloquium Series, University of Alabama.
\item Faulkenberry, T. J. (April 2021). A Mathematician's Apology: What a Life in Mathematics has Taught Me about Teaching Psychology,  Keynote Address - Southwestern Teachers of Psychology Conference (SWToP), San Antonio, TX.
\item Faulkenberry, T. J. (June 2020). Workshop: Bayesian statistics in numerical cognition. Invited workshop for the Math Cognition and Learning Society, Dublin, Ireland (cancelled due to COVID-19)
\item Faulkenberry, T. J. (June 2020), Workshop on Bayesian Statistics. Invited Lecture, University College - Dublin, Dublin, Ireland (cancelled due to COVID-19)
\item Faulkenberry, T. J. (April 2020). Developing an interactive web application for computing Bayes factors from summary statistics. Tarleton Psychological Sciences Day, Stephenville, TX.
\item Faulkenberry, T. J. (April 2020) A Mathematician's Apology: What a Life in Mathematics has Taught Me about Teaching Psychology,  Keynote Address - Southwestern Teachers of Psychology Conference (SWToP), Frisco, TX (cancelled due to COVID-19)
\item Faulkenberry, T. J. (September 2019). Workshop: Bayesian statistics with JASP -- Angelo State University, San Angelo, TX.
\item Faulkenberry, T. J. (June 2019). Workshop: Bayesian statistics in numerical cognition. Math Cognition and Learning Society, Ottawa, ON.
\item Faulkenberry, T. J. (August, 2018). Workshop on R and Bayesian Statistics -- Texas Lutheran University, Seguin, TX.
\item Faulkenberry, T. J. (April, 2018). Introduction to applied Bayesian hypothesis testing -- Faculty Research Coffee Hour, Stephenville, TX.
\item Faulkenberry, T. J. (December, 2017). Mental representations of two-digit numbers. Texas A\&M University - San Antonio Speakers' Series, San Antonio, TX.
\item Faulkenberry, T. J. (September, 2017). Modeling response times in mental arithmetic. Baylor University Psychology and Neuroscience Speaker Series, Waco, TX.
\item Faulkenberry, T. J. (April, 2017). The Pope, Bayes' Theorem, and Harry Potter: A statistical drama in three acts.  Tarleton Psychology Club, Stephenville, TX.
\item Faulkenberry, T. J. (March, 2017). Using mathematical modeling to understand mental arithmetic. Tarleton Math Club, Stephenville, TX.
\item Faulkenberry, T. J. (Nov. 2015). Associations between number and space in mental arithmetic.  Psychological Sciences Open House, Stephenville, TX.
\item Faulkenberry, T. J. et al. (Oct. 2015). Publishing in the digital age.  CII Panel Presentation, Stephenville, TX.
\item Faulkenberry, T. J. (June, 2015). Discussion of Marghetis et al. (2014). Carleton Math Cognition Lab, Ottawa, Ontario.
\item Smith, K. H., Riggs, B., Faulkenberry, E. D., \& Faulkenberry, T. J. (May, 2014). A snapshot of preservice teacher beliefs: A factor analytic method. Tarleton State University Math Day 2014.
\item Faulkenberry, T. J. (Feb, 2014). Detecting cognitive processes via the motions of the hand: Studies in numerical cognition.  Psychology \& Counseling Department Seminar, Tarleton State University.Math
\item Faulkenberry, T. J. (April, 2013). Estimating the working memory requirements of mental arithmetic. Mathematics Education Seminar, University of Texas - Arlington, Arlington, TX.
\item Faulkenberry, T. J. (April, 2012). Reconsidering the magic number 7: Measuring and modeling working memory capacity. Mathematics Department Colloquium, Southeastern Oklahoma State University, Durant, OK.
\item Faulkenberry, T. J. (May, 2012). Arctangent approximations of \(\pi\). Math Club Invited Speaker, Texas A\&M University - Commerce
\item Faulkenberry, T. J. (Feb, 2012). Reconsidering the magic number 7: Measuring and modeling working memory capacity. Mathematics Department Colloquium, Texas A\&M University - Commerce.
\item Faulkenberry, T. J. (2011). Introduction to \LaTeX{}, Mathematics Department Colloquium, Texas A\&M University - Commerce
\item Faulkenberry, T. J. (2009). Working memory in mathematical cognition: The case for fractions. Mathematics Department Colloquium, Texas A\& M University - Commerce.
\item Faulkenberry, T. J. (2007). Uses, mis-uses, and non-uses of probability and statistics. Math club invited lecture, Texas A\&M University - Commerce.
\item Faulkenberry, T. J. (2006). Continuous dynamics among phonological competitors. Cognitive Science Seminar, Texas A\&M University - Commerce.
\item Faulkenberry, T. J. (2006). The evolution of color language. Cognitive Science Seminar, Texas A\&M University - Commerce.
\item Faulkenberry, T. J. (2006). A computational model of event segmentation based on perceptual prediction. Cognitive Science Seminar, Texas A\&M University - Commerce.
\item Faulkenberry, T. J. (2006). An introduction to latent semantic analysis. Cognitive Science Seminar, Texas A\&M University - Commerce.
\item Faulkenberry, T. J. (2006). Dissections in mathematics. Math club invited lecture, Texas A\&M University - Commerce.
\item Faulkenberry, T. J. (2006). Embodied cognition: The role of body and mind in abstract thought. Mathematics Education Seminar, Texas A\&M University - Commerce.
\item Faulkenberry, T. J. (2005). A cognitive map for mathematical induction. Mathematics Education Seminar, Texas A\&M University - Commerce.
\item Faulkenberry, T. J. (2005). Reflective abstraction in advanced mathematical thinking. Mathematics Education Seminar, Texas A\&M University - Commerce.
\item Faulkenberry, T. J. (2005). Explorations in Flatland. Mathematics Colloquium, Texas A\&M University - Commerce.
\item Faulkenberry, T. J. (2005). What is mathematics education research? Mathematics Education Seminar, Texas A\&M University - Commerce.
\item Faulkenberry, T. J. (2004). Where do all the knots live: Templates and surface dynamics. Mathematics Colloquium, Texas A\&M University - Commerce.
\item Faulkenberry, T. J. (2003). A beginner’s guide to 3-manifolds. Graduate Student Colloquium, University of North Texas.
\item Faulkenberry, T. J. (2002). Determining the shape of space. Mathematics Colloquium, University of Central Oklahoma.
\item Faulkenberry, T. J. (2002). Determining the shape of space. Mathematics Colloquium, East Central University.
\item Faulkenberry, T. J. (2002). Algorithms in topology. Mathematics Colloquium, Southeastern Oklahoma State University.
\end{enumerate}

\subsection*{Research Funding}
\label{sec:orgbb25adf}

\emph{PI unless otherwise noted.  Total funding = \$558,374}

\begin{itemize}
\item 2023-2025, Tarleton State University, Postdoctoral Research Scholars Program, \$168,000. \emph{Bayesian modeling of individual differences in mathematical cognition}
\item 2022-2023, National Science Foundation / Mathematical Association of America, National Research Experiences for Undergraduates Program (NREUP), \$29,425. \emph{CMAT: Computational Mathematics at Tarleton}
\item 2022 (summer), Tarleton State University, President's Excellence in Research Scholars (PERS), \$29,993. \emph{Tarleton Researchers in Computational Mathematics}.
\item 2021-2022, National Science Foundation / Mathematical Association of America, National Research Experiences for Undergraduates Program (NREUP), \$30,125. \emph{CMAT: Computational Mathematics at Tarleton}
\item 2021 (summer), Tarleton State University, President's Excellence in Research Scholars (PERS), \$31,500. \emph{REU Site: Computational Mathematics at Tarleton}.
\item 2020-2021, Tarleton State University, Faculty-Student Research Grant, \$5000, \emph{Developing an interactive web-based calculator for Bayesian statistics}
\item 2019 (fall), Tarleton State University, Faculty Development Grant, \$1000, \emph{Travel: Math Cognition and Learning Society Conference in Dublin, Ireland}.
\item 2019 (summer), National Science Foundation / Mathematical Association of America, National Research Experiences for Undergraduates Program (NREUP), \$29,663. \emph{CMAT: Computational Mathematics at Tarleton}
\item 2018-2019, Tarleton State University, Faculty-Student Research Grant, \$5000, \emph{Modeling individual difference structures in numerical cognition}
\item 2018 (spring), Tarleton State University, OSRCA Undergrad. Res. Assistantship, \$1000, \emph{Using hierarchical Bayesian modeling to uncover the cognitive mechanisms underlying associations between number and space}
\item 2017 (fall), Tarleton State University, Faculty Development Grant, \$1000, \emph{Travel: Psychonomic Society Meeting in Vancouver, BC}
\item 2017 (summer), Tarleton State University, First Year Research Experience (FYRE), \$4000, \emph{Using the Wiener diffusion process to model response time distributions in a numerical decision task}
\item 2015 (fall), Tarleton State University, Faculty Development Grant, \$750, \emph{Travel: Psychonomic Society Meeting in Boston, MA}
\item 2016 (summer), Tarleton State University, First Year Research Experience (FYRE), \$6500, \emph{Investigating the dynamics of operator preview effects in mental arithmetic}
\item 2016 (summer), Tarleton State University, OSRCA Undergrad. Res. Assistantship, \$4000, \emph{Testing competing models of two-digit number representation: Decomposed, holistic, or hybrid?}
\item 2016 (spring), Tarleton State University, OSRCA Undergrad. Res. Assistantship, \$1000, \emph{Testing decomposed versus holistic fraction representations via an implicit priming task}
\item 2016 (spring), Tarleton State University, OSRCA Undergrad. Res. Assistantship, \$1000, \emph{Is memory "retrieval" in single digit arithmetic really just rapid shifts of attention along a mental number line?}
\item 2015 (fall), Tarleton State University, Faculty Development Grant, \$750, \emph{Travel: Psychonomic Society Meeting in Chicago, Illinois}
\item 2015 (fall), Tarleton State University, OSRCA Undergrad. Res. Assistantship, \$1000, \emph{The effects of numerical fluency on mental representations of two-digit numbers.}
\item 2015 (fall), Tarleton State University, OSRCA Undergrad. Res. Assistantship, \$1000, \emph{Spatial-numerical associations in mental arithmetic.}
\item 2015 (summer), Tarleton State University, First Year Research Experience (FYRE), \$6500, \emph{Mental representations of two-digit numbers}
\item 2015 (summer), Tarleton State University, OSRCA Undergrad. Res. Assistantship, \$3500, \emph{Spatial-numerical associations in mental arithmetic.}
\item 2015 (spring), Tarleton State University, OSRCA Undergrad. Res. Assistantship, \$1000, \emph{Are the stages of cognitive arithmetic additive or interactive? The effects of numerical surface form on an addition production task.}
\item 2015 (spring), Society for the Teaching of Psychology, Early Career Travel Grant, \$350, \emph{Travel: Southwestern Teachers of Psychology Conference in Wichita, Kansas}
\item 2014-2015, Tarleton State University, Organized Research Grant, \$9560, \emph{Investigating the cognitive factors behind mathematics learning disability}
\item 2014 (fall), Tarleton State University, OSRCA Undergrad. Res. Assistantship, \$1000, \emph{The effects of numerical surface form on strategies for mental arithmetic verification}
\item 2014 (fall), Tarleton State University, Faculty Development Grant, \$508, \emph{Travel: Psychonomic Society Meeting in Long Beach, California}
\item 2014 (summer), Tarleton State University, OSRCA Undergrad. Res. Assistantship, \$3500, \emph{Using hand tracking to analyze mental representations of fractions}
\item 2014 (spring), Tarleton State University, QEP Startup Grant, \$1500, \emph{Applied Learning Experience: Undergraduate Research in Mathematical Cognition}
\item 2013 (fall), Tarleton State University, Faculty Development Grant, \$630, \emph{Travel: Psychonomic Society Meeting in Toronto, Ontario}
\item 2012-2013, National Science Foundation: Robert Noyce Scholarship Program, \$174,020 (Co-PI with Ben Jang), \emph{Building the Capacity for Math and Science Teacher Training}
\item 2010, Texas A\&M University – Commerce, OSP Research Grant, \$5000, \emph{Mouse Tracking in Mathematical Cognition}
\item 2008, Texas A\&M University – Commerce, OSP Mini Grant, \$600, \emph{IoLab Button Box for Psyscope X}
\end{itemize}

\subsection*{Editorial roles}
\label{sec:orgcfa8706}
\begin{itemize}
\item Associate Editor (2024-present): \emph{Behavior Research Methods}
\item Guest Editor (2023-present): \emph{Canadian Journal of Experimental Psychology}
\item Editorial Board (2019-present): \emph{Journal of Numerical Cognition}
\item Associate Editor (2016-present): \emph{Journal of Psychological Inquiry}
\item Associate Editor (2017-2021): \emph{Journal of European Psychology Students}
\item Guest Editor (2016-2018): \emph{Journal of Numerical Cognition}
\item Associate Editor (2017-2019): \emph{Frontiers in Psychology: Cognition Section}
\item Review Editor (2016-2017): \emph{Frontiers in Psychology: Cognition Section}
\end{itemize}
\subsection*{Reviewing}
\label{sec:org58acc68}

\begin{itemize}
\item Ad hoc reviewer for the following journals: \emph{Acta Psychologica, Advances in Methods and Practices in the Psychological Sciences, Attention, Perception, \& Psychophysics, Behavior Research Methods, British Journal of Developmental Psychology, Canadian Journal of Experimental Psychology, Cognitive Processing, Cognition, Cognitive Science, Computational Brain and Behavior, Frontiers in Psychology, Indian Journal of Science and Technology, Journal of Cognitive Psychology, Journal of Experimental Child Psychology, Journal of Experimental Psychology: General, Journal of Numerical Cognition, Journal of Open Source Software, Learning and Individual Differences, Mathematical Population Studies, Mathematics Teacher, Mathematics Teaching in the Middle School, Meta-Psychology, PLOS One, Proceedings of the Research Council on Mathematics Learning, Psychological Methods, Psychonomic Bulletin and Review, Quarterly Journal of Experimental Psychology}
\item External Examiner
\begin{itemize}
\item 2017, Corinna Jones, Ph.D., University of Huddersfield, UK
\end{itemize}
\item Panelist/Reader
\begin{itemize}
\item 2017-2019, Judge, American Statistical Association Statistics Project Competition
\item 2015-2020, Reader, AP Statistics Exam, Kansas City, MO
\item 2012-2015, Panelist, National Science Foundation, Washington, DC
\end{itemize}
\item Grant proposal reviewer for National Science Foundation, Social Sciences and Humanities Research Council of Canada, National Science Centre (Poland)
\item Textbook reviewer for Psychology Press, Routledge, Sage, Taylor \& Francis, Cambridge University Press
\item Academic Program Reviewer
\begin{itemize}
\item 2023: B.A. in Psychology, Colorado Mesa University
\end{itemize}
\end{itemize}

\subsection*{Professional Memberships}
\label{sec:org7628226}
\begin{itemize}
\item American Mathematical Society
\item American Statistical Association
\item Mathematical Association of America
\item Mathematical Cognition and Learning Society
\item Psychonomic Society
\item Society for Mathematical Psychology
\item Southwestern Psychological Association (SWPA)
\end{itemize}

\subsection*{Professional Service}
\label{sec:orga87deb4}

National/regional service 

\begin{itemize}
\item Chair, Psychonomic Society Finance Commitee (appointed 2024-present)
\item Member, Psychonomic Society Finance Committee (appointed 2023)
\item President, Southwestern Psychological Association (elected 2021-2024)
\item Chair, ASA-MAA Joint Committee on Statistics and Data Science Education (appointed 2022)
\item Member, MAA Council on Teaching and Learning (appointed 2022)
\item Secretary/Treasurer, MAA Special Interest Group (SIGMAA) on Statistics Education (elected 2020-2021)
\item Member, ASA-MAA Joint Committee on Statistics Education (appointed 2020-2022)
\item Treasurer, Southwestern Psychological Association (appointed, 2017-2021)
\item Program Review Committee member, Psychonomic Society (appointed 2018-2021)
\item Texas Representative, Southwestern Psychological Association (elected, 2015-2017)
\item Steering Committee Member, Southwestern Teachers of Psychology (appointed, 2015-2016)
\item Nominating Committee Chair, Southwestern Psychological Association (appointed, 2015)
\item Advisory Board Member, Collaborative Replications and Education Project (CREP) (appointed, 2014-2016)
\item Session Chair, Psychonomics Annual Meeting (2014, 2015)
\item Session Chair, Southwestern Psychological Association Meeting (2015)
\item Conference Committee Member, Research Council on Mathematics Learning (elected, 2012-2015)
\end{itemize}

University service

\begin{itemize}
\item Chair, Institutional Review Board (IRB) (2018-2023)
\item Member, General Education and Academic Assessment Committee (2019-present)
\item Faculty Research Fellow (2018-2019)
\item Official Advisor for Alpha Chi Honor Society, Tarleton (2015-present)
\item State Non-Funded Course Review Group, Member, Tarleton (2015-2019)
\item University Research Committee, Member, Tarleton (2015-2021)
\item Student Research and Creative Activity Advisory Committee, Member, Tarleton (2013-2021)
\item Member, Institutional Review Board (IRB) (2017-2018)
\item Faculty Fellow, Tarleton (2016-2018)
\item Honors Advisory Committee for College of Education, Member, Tarleton (2015-2017)
\item Session Chair and Judge, TAMUS Pathways Symposium (2017)
\item University ALE Task Force, Member, Tarleton (2016-2017)
\item Curriculum Committee, College of Education, Member, Tarleton (2014-2017)
\item Greater Expectations Task Force, Member, Tarleton (2014-2015)
\item Student/Faculty Marshall for Commencement, Tarleton (many times)
\item Session Chair and Judge, Tarleton Research Symposium (2014, 2015)
\item External Search Committee member, Department of Engineering, TAMU-C (2012-2013)
\item University Honors Council, Member, TAMU-C (2012-2013)
\item Liberal Studies Committee, Member, TAMU-C (2012-2013)
\item Developmental Appeals Committee, Member, TAMU-C (2010-2012)
\end{itemize}

Department service

\begin{itemize}
\item Human subjects research pool coordinator, Tarleton (2018-2022)
\item Organized \emph{Psychological Sciences Day}, Tarleton (2017-2020)
\item Search Committee Chair, Department of Psychological Sciences, Tarleton (2014, 2015, 2020)
\item Texan Orientation, Department of Psychological Sciences, Tarleton (2014-2018)
\item Texan Tour Speaker, Department of Psychological Sciences, Tarleton (2015)
\item Organizer, Psychology Department Seminar, Tarleton (2013-2014)
\item Psychology Scholarship Committee, Tarleton (2013-2015)
\item Psychology Undergraduate Programs Committee, Member, TAMU-C (2011-2012)
\end{itemize}

\subsection*{Courses Taught}
\label{sec:orgc1e8a73}

Tarleton State University

\begin{itemize}
\item PSYC 2301: General Psychology (Honors), Fall 13,14,15
\item PSYC 2317: Statistical Methods for Psychology, Fall 19,20,21,23; Sp 20
\item PSYC 2301: General Psychology, Summer 14,15
\item PSYC 3301: Psychology of Learning, Fall 13,14,15,16
\item PSYC 3303: Educational Psychology, Fall 13; Spring 14; Summer 14,15,16
\item PSYC 3309: Writing in Psychology, Spring 16
\item PSYC 3320: Psycholinguistics, Summer 17,18,19,20,21
\item PSYC 3330: Elem Statistics for Behav Science, Fall 14,15,16,17,18; Sp 15,16,17,18; Su 16,17,18
\item HONS 3385: Honors Seminar (Numerical Cognition), Spring 15
\item PSYC 3435: Prin Research for Behav Science, Fall 14,15,16,17,18,19,20; Sp 15,16,17,18,19,21; Su 15,16,17
\item PSYC 4301: Psychological Tests and Measurements, Sp 20,21
\item PSYC 4386: Advanced Statistical Methods, Spring 14
\item PSYC 4386: Methods in Experimental Psychology, Spring 15; Fall 15
\item PSYC 4386: Problems in Numerical Cognition, Fall 15
\item PSYC 5303: Theories of Learning, Fall 16,22
\item PSYC 5301: Research Methods, Spring 14,15,17,18,19,21
\item PSYC 5304: Human Development, Spring 14
\item PSYC 5316: Advanced Quantitative Methods, Fall 17,18,19,20,21,22,23
\item PSYC 5322: Psychometrics, Sp 20
\item PSYC 5379: Advanced Psycholinguistics, Summer 17,18,19,20,21
\item EDAD 6313: Statistical Methods for Educational Leadership, Spring 16
\end{itemize}

\subsection*{Student Mentoring}
\label{sec:orga0757eb}
\subsubsection*{Masters Thesis Chair}
\label{sec:org48c0243}
\begin{itemize}
\item Steven McMullin (Applied Psychology, Tarleton, in progress)
\item Keelyn Brennan (Applied Psychology, Tarleton, graduated 2023)
\item Bryanna Scheuler (Applied Psychology, Tarleton, graduated 2022)
\item Mihaela Codreanu (Applied Psychology, Tarleton, graduated 2022)
\item Annie Lenoir (Applied Psychology, Tarleton, graduated 2021)
\item Kristen Bowman (Applied Psychology, Tarleton, graduated 2020)
\item Chelsea Bradley (Applied Psychology, Tarleton, graduated 2018)
\end{itemize}

\subsubsection*{Doctoral Committees}
\label{sec:orgcdb5b6c}
\begin{itemize}
\item Angelika Stefan (Psychological Methods, University of Amsterdam, graduated 2023)
\item Jessica Cervantes (Educational Leadership, Tarleton, graduated 2021)
\item Jeni McNeely (Educational Leadership, Tarleton, graduated 2016)
\item Trina Geye (Psychology, TAMU-C, graduated 2016)
\item Beth Nikopoulous (Psychology, TAMU-C, graduated 2015)
\item Donna Peters (Psychology, TAMU-C, graduated 2013)
\end{itemize}

\subsubsection*{Masters Committees}
\label{sec:org0c4cf70}
\begin{itemize}
\item Shanna Coury (Psychology, TAMU-C, in progress)
\item Simon Rook (Applied Psychology, Tarleton, graduated 2022)
\item Rene Wallace (Applied Psychology, Tarleton, graduated 2021)
\item Kayli Colpitts (Applied Psychology, Tarleton, graduated 2020)
\item Kody Lamb (Applied Psychology, Tarleton, graduated 2018)
\item Trina Geye (Psychology, TAMU-C, graduated 2015)
\item Beth Nikopoulous (Psychology, TAMU-C, graduated 2013)
\item Heather Oetker (Special Education, TAMU-C, graduated 2012)
\item Joshua Patterson (Mathematics, TAMU-C, graduated 2011)
\end{itemize}

\subsubsection*{Honors Thesis Chair}
\label{sec:org2b85d42}
\begin{itemize}
\item Bella Zapata (Psychology, Tarleton, graduated 2023), \emph{Closed form methods for estimating parameters of response time distributions}
\item Kristen Bowman (Psychology, Tarleton, graduated 2018), \emph{Nonwords induce reverse priming effects in a lexical decision task}
\item Anissa Eid (Psychology, Tarleton, graduated 2018), \emph{Cognitive mechanisms underlying spatial-numerical associations}
\item Paige Woodard (Psychology, Tarleton, graduated 2017), \emph{Mental arithmetic: Relationship between encoding and calculation processes}
\item Sarah Montgomery (High Honors in Psychology, TAMU-C, graduated 2013), \emph{Measuring the Working Memory Requirements of Mental Arithmetic}
\item Emily Dalton (Honors in Psychology, TAMU-C, graduated 2013), \emph{The Effects of Generation on False Memory for Numbers}
\item Kaytlin Reid (Honors in Interdisciplinary Studies, TAMU-C, graduated 2013), \emph{The Role of Working Memory in Mental Fraction Computation}
\item Douglas Boney (Honors in Mathematics, TAMU-C, graduated 2013), \emph{Knot Polynomials}
\item Samantha Reece (Honors in Sociology, TAMU-C, graduated 2012), \emph{The Effects of Stereotype Threat on Cheating Behavior in Mathematics}
\end{itemize}

\subsubsection*{Honors Thesis Committees}
\label{sec:org52ce467}

\begin{itemize}
\item Carmen Phelps (English, TAMU-C, 2013)
\item Morgan Lutz (Psychology, TAMU-C, 2013)
\item Nick Bredberg (Physics, TAMU-C, 2012)
\item Kallie Hinton (Mathematics Education, TAMU-C, 2011)
\item Lindsey Preston (Mathematics Education, TAMU-C, 2011)
\end{itemize}
\end{document}