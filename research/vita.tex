% Created 2017-07-24 Mon 11:53
\documentclass[article,10pt]{article}
\usepackage[utf8]{inputenc}
\usepackage[T1]{fontenc}
\usepackage{fixltx2e}
\usepackage{graphicx}
\usepackage{longtable}
\usepackage{float}
\usepackage{wrapfig}
\usepackage{rotating}
\usepackage[normalem]{ulem}
\usepackage{amsmath}
\usepackage{textcomp}
\usepackage{marvosym}
\usepackage{wasysym}
\usepackage{amssymb}
\usepackage{hyperref}
\tolerance=1000
\usepackage[left=1in,right=1in,bottom=1in,top=1in]{geometry}
\usepackage{fancyhdr}
\pagestyle{fancyplain}
\lfoot{Last updated \today} \cfoot{} \rfoot{\thepage}
\date{\today}
\title{}
\hypersetup{
  pdfkeywords={},
  pdfsubject={},
  pdfcreator={Emacs 25.2.1 (Org mode 8.2.10)}}
\begin{document}


\section*{Thomas J. Faulkenberry, Ph.D.}
\label{sec-1}

Department of Psychological Sciences\\
Tarleton State University\\
Box T-0820, Stephenville, TX 76402\\
Phone: +1 (254) 968-9816\\
Email: faulkenberry@tarleton.edu\\
Website: \url{http://tomfaulkenberry.github.io}

\subsection*{Education}
\label{sec-1-1}
\begin{itemize}
\item Ph.D., Psychology, Texas A\&M University – Commerce, 2010
\item M.S., Mathematics, Oklahoma State University - 2002
\item B.S. with highest honors, Mathematics, Southeastern Oklahoma State University, 2000
\end{itemize}

\subsection*{Academic positions}
\label{sec-1-2}
\begin{itemize}
\item 2013-present: Assistant Professor, Department of Psychological Sciences, Tarleton State University
\item 2012-2013: Assistant Professor, Department of Mathematics, Texas A\&M University – Commerce
\item 2010-2012: Visiting Assistant Professor, Department of Psychology and Special Education, Texas A\&M University – Commerce
\item 2005-2010: Lecturer, Department of Mathematics, Texas A\&M University - Commerce
\end{itemize}

\subsection*{Honors and awards}
\label{sec-1-3}
\begin{itemize}
\item Faculty Excellence in Scholarship Award, Tarleton State University College of Education, 2016
\item President's All-Purple Award, Tarleton State University, 2016
\item Nominee, Bruce N. Chaloux Award for Early Career Excellence in Online Education, Sloan Consortium, 2015
\item Fellow, The Psychonomic Society, 2014
\item Texas A\&M University System Teaching Excellence Award, 2010-2011
\item Best Graduate Student Presentation, Annual Research Symposium, TAMU-C, 2010
\item O.H. Hamilton Fellowship in Mathematics, Oklahoma State University, 2002
\item S. J. Scroggs Distinguished Graduate Fellowship, Oklahoma State University, 2001
\item Regional University Scholarship (4 years), Southeastern Oklahoma State University, 1996
\end{itemize}

\subsection*{Administrative activities}
\label{sec-1-4}
\begin{itemize}
\item 2011-2012: Director, Center for Undergraduate Research and Creative Activities, TAMU-C
\item 2010-2011: Director, Math \& Science Teacher Preparation Program (LeoTEACH), TAMU-C
\end{itemize}

\subsection*{Research publications}
\label{sec-1-5}
\begin{enumerate}
\item Frampton, A. R., \& Faulkenberry, T. J. (in press). Mental arithmetic processes: Testing the independence of encoding and calculation. To appear in \emph{Journal of Psychological Inquiry}.
\item Faulkenberry, T. J. (2017). A single-boundary accumulator model of repsonse times in an arithmetic verification task. \emph{Frontiers in Psychology}, \emph{8:1225}. doi: \href{http://dx.doi.org/10.3389/fpsyg.2017.01225}{10.3389/fpsyg.2017.01225/}
\item Faulkenberry, T. J., Cruise, A., \& Shaki, S. (2017). Reversing the manual digit bias in two-digit number comparison. \emph{Experimental Psychology}, \emph{64}, 191-204.  doi: \href{http://dx.doi.org/10.1027/1618-3169/a000365}{10.1027/1618-3169/a000365}
\item Sobel, K. V., Puri, A. M., Faulkenberry, T. J., \& Dague, T. D. (2017). Visual search for conjunctions of physical and numerical size shows that they are processed independently. \emph{Journal of Experimental Psychology: Human Perception \& Performance}, \emph{43}, 444-453. doi: \href{http://dx.doi.org/10.1037/xhp0000323}{10.1037/xhp0000323}
\item Faulkenberry, T. J., \& Tummolini, L. (2016). Commentary: Is there any Influence of Variations in Context on Object-Affordance Effects in Schizophrenia? Perception of Property and Goals of Action). \emph{Frontiers in Psychology}, \emph{7:1915}. doi: \href{http://dx.doi.org/10.3389/fpsyg.2016.01915}{10.3389/fpsyg.2016.01915}
\item Faulkenberry, T. J. (2016). Testing a direct mapping versus competition account of response dynamics in number comparison. \emph{Journal of Cognitive Psychology}, \emph{28}, 825-842. doi: \href{http://dx.doi.org/10.1080/20445911.2016.1191504}{10.1080/20445911.2016.1191504}
\item Sobel, K. V., Puri, A. M., \& Faulkenberry, T. J. (2016). Bottom-up and top-down attentional contributions to the size-congruity effect. \emph{Attention, Perception, \& Psychophysics}, \emph{78}, 1324-1336. doi: \href{http://dx.doi.org/10.3758/s13414-016-1098-3}{10.3758/s13414-016-1098-3}
\item Faulkenberry, T. J., Cruise, A., Lavro, D., \& Shaki, S. (2016). Response trajectories capture the continuous dynamics of the size congruity effect. \emph{Acta Psychologica}, \emph{163}, 114-123. doi: \href{http://dx.doi.org/10.1016/j.actpsy.2015.11.010}{10.1016/j.actpsy.2015.11.010}
\item Faulkenberry, T. J., Montgomery, S. A., \& Tennes, S. N. (2015). Response trajectories reveal the temporal dynamics of fraction representations. \emph{Acta Psychologica}, \emph{159}, 100-107. doi: \href{http://dx.doi.org/10.1016/j.actpsy.2015.05.013}{10.1016/j.actpsy.2015.05.013}
\item Faulkenberry, T. J., \& Rey, A. R. (2014). Extending the reach of mousetracking in numerical cognition: A comment on Fischer and Hartmann (2014). \emph{Frontiers in Psychology}, \emph{5}:1436. doi: \href{http://dx.doi.org/10.3389/fpsyg.2014.01436}{10.3389/fpsyg.2014.01436}
\item Faulkenberry, T. J. (2014). Hand movements reflect competitive processing in numerical cognition. \emph{Canadian Journal of Experimental Psychology}, \emph{68}, 147-151. doi: \href{http://dx.doi.org/10.1037/cep0000021}{10.1037/cep0000021}
\item Faulkenberry, T. J., \& Geye, T. L. (2014). The cognitive origins of mathematics learning disability: A review. \emph{The Rehabilitation Professional}, \emph{22} (1), 9-16.
\item Faulkenberry, T. J., \& Faulkenberry, E. D. (2013). Teaching integer arithmetic without rules: An embodied approach. \emph{Oklahoma Journal of School Mathematics}, \emph{5} (2), 5-14.
\item Faulkenberry, T. J., (2013). The conceptual/procedural distinction belongs to strategies, not tasks: A comment on Gabriel et al. (2013). \emph{Frontiers in Psychology}, \emph{4}:820. doi: \href{http://dx.doi.org/10.3389/fpsyg.2013.00820}{10.3389/fpsyg.2013.00820}
\item Faulkenberry, T. J., \& Montgomery, S. A. (2013). The primacy of fraction components in adults’ numerical judgements. In Reeder, S. L. and Matney, G. T. (Eds.). \emph{Proceedings of the 40th Annual Meeting of the Research Council on Mathematics Learning} (pp. 155-162). Tulsa, OK: RCML
\item Faulkenberry, T. J. (2013). How the hand mirrors the mind: The embodiment of numerical cognition. In Reeder, S. L. and Matney, G. T. (Eds.). \emph{Proceedings of the 40th Annual Meeting of the Research Council on Mathematics Learning} (pp. 205-212). Tulsa, OK: RCML
\item Faulkenberry, E. D., \& Faulkenberry, T. J. (2012). Do you see what I see? An exploration of self-perception in the classroom. In S. L. Reeder (Ed.), \emph{Proceedings of the 39th Annual Meeting of the Research Council on Mathematics Learning} (pp. 121-126). Charlotte, NC: RCML.
\item Faulkenberry, T. J., \& Pierce, B. H. (2011). Mental representations in fraction comparison: Holistic versus component-based strategies. \emph{Experimental Psychology}, \emph{58}, 480-489. doi: \href{http://dx.doi.org/10.1027/1618-3169/a000116}{10.1027/1618-3169/a000116}
\item Faulkenberry, T. J. (2011). Individual differences in mental representations of fraction magnitude. In S. Reeder (Ed.) \emph{Proceedings of the 38th Annual Meeting of the Research Council on Mathematics Learning} (pp. 136-143). Cincinnati, OH: RCML.
\item Faulkenberry, E. D., \& Faulkenberry, T. J. (2010). Transforming the way we teach function transformations. \emph{Mathematics Teacher}, \emph{104}, 29-33.
\item Faulkenberry, T. J. (2010). The working memory demands of simple fraction strate- gies. In S. Reeder (Ed.) \emph{Proceedings of the 37th Annual Meeting of the Research Council on Mathematics Learning} (pp. 84-89). Conway, AR: RCML.
\item Faulkenberry, E. D. \& Faulkenberry, T. J. (2006). Constructivism in mathematics education: A historical and personal perspective. \emph{The Texas Science Teacher}, \emph{35}, 17- 22.
\end{enumerate}

\subsection*{Abstracts, columns, and book reviews}
\label{sec-1-6}
\begin{enumerate}
\item Faulkenberry, T. J. (2016). Motor dynamics support a competition model of number processing. \emph{Abstracts of the Psychonomic Society}, \emph{21}, 26.
\item Bowman, K. A., \& Faulkenberry, T. J. (2016). Testing competing models of two-digit number representation: Decomposed versus holistic processing. \emph{Abstracts of the Psychonomic Society}, \emph{21}, 285.
\item Faulkenberry, T. J. (2016). Decoding the development of mathematical thinking: A book review of \emph{Development of Mathematical Thinking: Neural Substrates and Genetic Influences}. \emph{PsycCRITIQUES}, \emph{61} (31). doi: \href{http://dx.doi.org/10.1037/a0040434}{10.1037/a0040434}
\item Faulkenberry, T. J. (2016). Undergraduate students: An endangered resource? \emph{SWPA Newsletter}
\item Faulkenberry, T. J., Cruise, A., Lavro, D., \& Shaki, S. (2015). Response trajectories support a late-interaction model of the size-congruity effect. \emph{Canadian Journal of Experimental Psychology, 69}, 346.
\item Faulkenberry, T. J., Cruise, A., \& Shaki, S. (2015). Reversing the manual decade bias in two-digit number comparison. \emph{Abstracts of the Psychonomic Society, 20}, 39.
\item Geye, T. L, \& Faulkenberry, T. J. (2015). Response trajectories capture individual differences in a size congruity task. \emph{Abstracts of the Psychonomic Society, 20}, 249.
\item Faulkenberry, T. J., Cruise, A., Lavro, D., \& Shaki, S. (2014). Response trajectories capture the continuous dynamics of the size-congruity effect. \emph{Abstracts of the Psychonomic Society, 19}, 53.
\item Faulkenberry, T. J. (2013). Measuring the working memory requirements of mental arithmetic. \emph{Canadian Journal of Experimental Psychology, 67}, 281.
\item Faulkenberry, T. J. (2013). Measuring the working memory requirements of mental arithmetic. \emph{Abstracts of the Psychonomic Society, 18}, 203-204.
\item Faulkenberry, T. J. (2012). The temporal dynamics of fraction representations: Components are processed first. \emph{Canadian Journal of Experimental Psychology, 66}, 310.
\item Faulkenberry, T. J. \& Montgomery, S. A. (2012). The primacy of components in numerical fractions. \emph{Abstracts of the Psychonomic Society, 17}, 206.
\item Faulkenberry, T. J. (2011). Brain-based mathematics: Promising practice or hopeful hype? \emph{RCML Intersection Points, 35} (3), 9-10.
\item Faulkenberry, T. J. \& Kelsey, A. R. (2011). Working memory and strategic performance in fraction comparison. \emph{Canadian Journal of Experimental Psychology, 65}, 311-311.
\item Faulkenberry, T. J. (2011). The dynamics of the SNARC effect: Evidence from mouse tracking. \emph{Canadian Journal of Experimental Psychology, 65}, 316-316.
\item Faulkenberry, T. J. (2011). Motor dynamics in numerical representations: Evidence from mouse tracking. \emph{Abstracts of the Psychonomic Society, 16}, 76-76.
\item Faulkenberry, T. J. (2010). The roles of phonological and visuo-spatial working memory resources in simple fraction strategies. \emph{Canadian Journal of Experimental Psychology, 64}, 302-302.
\item Lu, S. Wakefield, L. \& Faulkenberry, T. J. (2006). The roles of beginnings, overlap, and ends in event temporal relations. \emph{Abstracts of the Psychonomic Society, 11}, 9-9.
\end{enumerate}

\subsection*{Conference Presentations}
\label{sec-1-7}
\begin{enumerate}
\item Faulkenberry, T. J. (April, 2017). Accumulator models of decision processes in mental arithmetic. Southwestern Psychological Association, San Antonio, TX
\item Faulkenberry, T. J., \& Wood, J. (April, 2017). A Bayesian perspective on the operator preview paradigm in mental arithmetic. Southwestern Psychological Association, San Antonio, TX
\item Nejman, J., \& Faulkenberry, T. J. (April, 2017). Implicit priming reveals both holistic and decomposed processing in fraction comparison. Southwestern Psychological Association, San Antonio, TX
\item Wood, J., \& Faulkenberry, T. J. (April, 2017). The dynamics of operator preview effects in mental arithmetic. Southwestern Psychological Association, San Antonio, TX
\item Bowman, K., \& Faulkenberry, T. J. (April, 2017). Testing competing models of two-digit number representation: Decomposed versus holistic processing. Southwestern Psychological Association, San Antonio, TX
\item Faulkenberry, T. J. (April, 2016). Testing two accounts of response dynamics in a number comparison task. Southwestern Psychological Association, Dallas, TX
\item Faulkenberry, T. J. (April, 2016). Recent developments on the size congruity effect in numerical cognition. Southwestern Psychological Association, Dallas, TX
\item Rutledge, M., \& Faulkenberry, T. J. (April, 2016). Spatial-numerical associations in mental arithmetic. Southwestern Psychological Association, Dallas, TX
\item Geye, T., \& Faulkenberry, T. J. (April, 2016). Computer mousetracking reveals individual differences in a size congruity task. Southwestern Psychological Association, Dallas, TX
\item Bowman, K. A., \& Faulkenberry, T. J. (April, 2016). The effects of mathematical fluency on multi-digit number representations. Southwestern Psychological Association, Dallas, TX
\item Faulkenberry, T. J. (October, 2015). Testing a direct-mapping versus competition account of response dynamics in a number comparison task. ARMADILLO 2015, Waco, TX.
\item Bowman, K. A., \& Faulkenberry, T. J. (October, 2015). The effects of mathematical fluency on multi-digit number representations. ARMADILLO 2015, Waco, TX.
\item Bowman, K. A., \& Faulkenberry, T. J. (October, 2015). The effects of mathematical fluency on multi-digit number representations. TAMUS Pathways Symposium, Corpus Christi, TX.
\item Bowman, K. A., \& Faulkenberry, T. J. (October, 2015). The effects of mathematical fluency on multi-digit number representations. Tarleton Research Symposium, Stephenville, TX.
\item Faulkenberry, T. J. (April, 2015). Class-sourcing replications of reaction time studies: An example in mathematical cognition. Southwestern Teachers of Psychology Conference, Wichita, KS.
\item Geye, T., Fleming, B., \& Faulkenberry, T. J. (April, 2015). Validation of the calculation fluency test for measuring arithmetic skills. Southwestern Psychological Association, Wichita, KS.
\item Frampton, A., \& Faulkenberry, T. J. (April, 2015). Cognitive arithmetic processs: The effects of problem size and format on performance. Southwestern Psychological Association, Wichita, KS.
\item Faulkenberry, T. J. (April, 2015). Evidence for a late-interactions model of the numerical size congruity effect. Southwestern Psychological Association, Wichita, KS.
\item Harris Bozer, A., \& Faulkenberry, T. J. (April, 2015). Applying the CREATE pedagogical tool to the online animal behavior course to enhance scientific literacy.  2015 CIRTL Forum: Preparing the Future STEM Faculty for the Rapidly Changing Landscape of Higher Education, College Station, TX.
\item Frampton, A., \& Faulkenberry, T. J. (March, 2015). Cognitive arithmetic processes: The effects of numerical surface form on strategy choice. Texas Undergraduate Research Day at the Capitol, Austin, TX.
\item Faulkenberry, E. D., Smith, K., Riggs, E., \& Faulkenberry, T. J. (February, 2015). The evolution of PST’s beliefs: Examining the effect of teacher preparation. Research Council on Mathematics Learning, Las Vegas, NV.
\item Faulkenberry, T. J. (October, 2014).  Hand movements reflect competitive processing in a numerical parity task. ARMADILLO 2014, Norman, OK.
\item Faulkenberry, T. J. (October, 2014). The dynamics of fraction representations: Components are processed first. ARMADILLO 2014, Norman, OK.
\item Faulkenberry, T. J. (April, 2014). Hand movements reflect competitive processing in numerical fraction representations. Southwestern Psychological Association, San Antonio, TX.
\item Faulkenberry, T. J. (April, 2014). A brief introduction to using R for teaching statistical methods. Southwestern Teachers of Psychology Conference, San Antonio, TX.
\item Faulkenberry, T. J. (March, 2014). A classroom activity for demonstrating confirmation bias. Tarleton Excellence in Teaching Conference, Stephenville, TX.
\item Smith, K. H., Riggs, B., Faulkenberry, E. D., \& Faulkenberry, T. J. (February, 2014). A snapshot of preservice teacher beliefs: A factor analytic method. Research Council on Mathematics Learning, San Antonio, TX.
\item Faulkenberry, T. J. (April, 2013). Modeling the roles of working memory and strategy type in fraction comparison. TX Section MAA Meeting, Texas Tech University, Lubbock, TX.
\item Faulkenberry, T. J. (March, 2013). Estimating the working memory requirements of mental arithmetic. OK-AR Section MAA Meeting, Oklahoma State University, Stillwater, OK.
\item Faulkenberry, T. J. (April, 2012). Some limitations in measuring working memory capacity. TX Section MAA Meeting, El Centro College, Dallas, TX.
\item Faulkenberry, T. J. (February, 2012). Examining the role of testing in learning mathematics: Directions for future research. 39th Annual Meeting of the Research Council on Mathematics Learning, Charlotte, NC.
\item Faulkenberry, T. J. \& Pierce, B. H. (October, 2011). The roles of working memory and strategy type in fraction comparison. ARMADILLO 2011, Commerce, TX.
\item Faulkenberry, T. J. (April, 2010). Working memory and strategy execution in simple fraction strategies. Annual Research Symposium, Texas A\& M University - Commerce.
\item Faulkenberry, T. J. (April, 2009). Mathematics anxiety among elementary education majors: Does test format matter?. Annual Research Symposium, Texas A\& M University - Commerce.
\item Faulkenberry, T. J. (February, 2009). Mathematics anxiety among elementary education majors. 36th Annual Meeting of the Research Council on Mathematics Learning, Rome, GA.
\item Faulkenberry, E. D. \& Faulkenberry, T. J. (February, 2008). An assessment of the mathematical knowledge of elementary preservice teachers with regard to number and operation. 35th Annual Meeting of the Research Council on Mathematics Learning, Oklahoma City, OK.
\item Faulkenberry, T. J. (February, 2008). Working memory: Cognitive and instructional implications for mathematics. 35th Annual Meeting of the Research Council on Mathematics Learning, Oklahoma City, OK.
\item Faulkenberry, E. D. \& Faulkenberry, T. J. (October, 2005). Using the geometry module in Teacher Quality grants. Charles A. Dana Center Higher Education Mathematics Conference, Austin, TX.
\item Faulkenberry, T. J. (April, 2005). Cognitive frameworks in advanced mathematical thinking. MAA Texas Section Meeting, University of Texas - Arlington.
\item Faulkenberry, T. J. (April, 2004). The shapes of 2-dimensional manifolds. MAA Texas Section Meeting, Texas A\&M University - Corpus Christi.
\item Faulkenberry, T. J. (March, 2003). Conway’s ZIP proof. MAA Oklahoma/Arkansas Section Meeting, University of Tulsa.
\item Faulkenberry, T. J. (March, 2002). Knot algorithms and their computational complexity. MAA Oklahoma/Arkansas Section Meeting, Henderson State University.
\item Faulkenberry, T. J. (March, 2002). Topology in the high school? National Council of Teachers of Mathematics Regional Conference, Oklahoma City, OK.
\item Faulkenberry, T. J. (March, 1999). The construction of a Riemann surface structure on a once-punctured torus. MAA Oklahoma/Arkansas Section Meeting, Arkansas Tech University.
\item Faulkenberry, T. J. (March, 1998). The classification of Markoff numbers on a once-punctured torus. MAA Oklahoma/Arkansas Section Meeting, Southern Nazarene University.
\end{enumerate}

\subsection*{Seminars and Invited Talks}
\label{sec-1-8}
\begin{enumerate}
\item Faulkenberry, T. J. (April, 2017). The Pope, Bayes' Theorem, and Harry Potter: A statistical drama in three acts.  Tarleton Psychology Club, Stephenville, TX.
\item Faulkenberry, T. J. (March, 2017). Using mathematical modeling to understand mental arithmetic. Tarleton Math Club, Stephenville, TX.
\item Faulkenberry, T. J. (Nov. 2015). Associations between number and space in mental arithmetic.  Psychological Sciences Open House, Stephenville, TX.
\item Faulkenberry, T. J. et al. (Oct. 2015). Publishing in the digital age.  CII Panel Presentation, Stephenville, TX.
\item Faulkenberry, T. J. (June, 2015). Discussion of Marghetis et al. (2014). Carleton Math Cognition Lab, Ottawa, Ontario.
\item Smith, K. H., Riggs, B., Faulkenberry, E. D., \& Faulkenberry, T. J. (May, 2014). A snapshot of preservice teacher beliefs: A factor analytic method. Tarleton State University Math Day 2014.
\item Faulkenberry, T. J. (Feb, 2014). Detecting cognitive processes via the motions of the hand: Studies in numerical cognition.  Psychology \& Counseling Department Seminar, Tarleton State University.Math
\item Faulkenberry, T. J. (April, 2013). Estimating the working memory requirements of mental arithmetic. Mathematics Education Seminar, University of Texas - Arlington, Arlington, TX.
\item Faulkenberry, T. J. (April, 2012). Reconsidering the magic number 7: Measuring and modeling working memory capacity. Mathematics Department Colloquium, Southeastern Oklahoma State University, Durant, OK.
\item Faulkenberry, T. J. (May, 2012). Arctangent approximations of $\pi$. Math Club Invited Speaker, Texas A\&M University - Commerce
\item Faulkenberry, T. J. (Feb, 2012). Reconsidering the magic number 7: Measuring and modeling working memory capacity. Mathematics Department Colloquium, Texas A\&M University - Commerce.
\item Faulkenberry, T. J. (2011). Introduction to \LaTeX{}, Mathematics Department Colloquium, Texas A\&M University - Commerce
\item Faulkenberry, T. J. (2009). Working memory in mathematical cognition: The case for fractions. Mathematics Department Colloquium, Texas A\& M University - Commerce.
\item Faulkenberry, T. J. (2007). Uses, mis-uses, and non-uses of probability and statistics. Math club invited lecture, Texas A\&M University - Commerce.
\item Faulkenberry, T. J. (2006). Continuous dynamics among phonological competitors. Cognitive Science Seminar, Texas A\&M University - Commerce.
\item Faulkenberry, T. J. (2006). The evolution of color language. Cognitive Science Seminar, Texas A\&M University - Commerce.
\item Faulkenberry, T. J. (2006). A computational model of event segmentation based on perceptual prediction. Cognitive Science Seminar, Texas A\&M University - Commerce.
\item Faulkenberry, T. J. (2006). An introduction to latent semantic analysis. Cognitive Science Seminar, Texas A\&M University - Commerce.
\item Faulkenberry, T. J. (2006). Dissections in mathematics. Math club invited lecture, Texas A\&M University - Commerce.
\item Faulkenberry, T. J. (2006). Embodied cognition: The role of body and mind in abstract thought. Mathematics Education Seminar, Texas A\&M University - Commerce.
\item Faulkenberry, T. J. (2005). A cognitive map for mathematical induction. Mathematics Education Seminar, Texas A\&M University - Commerce.
\item Faulkenberry, T. J. (2005). Reflective abstraction in advanced mathematical thinking. Mathematics Education Seminar, Texas A\&M University - Commerce.
\item Faulkenberry, T. J. (2005). Explorations in Flatland. Mathematics Colloquium, Texas A\&M University - Commerce.
\item Faulkenberry, T. J. (2005). What is mathematics education research? Mathematics Education Seminar, Texas A\&M University - Commerce.
\item Faulkenberry, T. J. (2004). Where do all the knots live: Templates and surface dynamics. Mathematics Colloquium, Texas A\&M University - Commerce.
\item Faulkenberry, T. J. (2003). A beginner’s guide to 3-manifolds. Graduate Student Colloquium, University of North Texas.
\item Faulkenberry, T. J. (2002). Determining the shape of space. Mathematics Colloquium, University of Central Oklahoma.
\item Faulkenberry, T. J. (2002). Determining the shape of space. Mathematics Colloquium, East Central University.
\item Faulkenberry, T. J. (2002). Algorithms in topology. Mathematics Colloquium, Southeastern Oklahoma State University.
\end{enumerate}

\subsection*{Research Funding}
\label{sec-1-9}

\emph{PI unless otherwise noted.  Total funding = \$216,418.}

\begin{itemize}
\item 2017 (summer), Tarleton State University, First Year Research Experience (FYRE), \$4000, \emph{Using the Wiener diffusion process to model response time distributions in a numerical decision task}
\item 2016 (summer), Tarleton State University, First Year Research Experience (FYRE), \$6500, \emph{Investigating the dynamics of operator preview effects in mental arithmetic}
\item 2016 (summer), Tarleton State University, OSRCA Undergrad. Res. Assistantship, \$4000, \emph{Testing competing models of two-digit number representation: Decomposed, holistic, or hybrid?}
\item 2016 (spring), Tarleton State University, OSRCA Undergrad. Res. Assistantship, \$1000, \emph{Testing decomposed versus holistic fraction representations via an implicit priming task}
\item 2016 (spring), Tarleton State University, OSRCA Undergrad. Res. Assistantship, \$1000, \emph{Is memory "retrieval" in single digit arithmetic really just rapid shifts of attention along a mental number line?}
\item 2015 (fall), Tarleton State University, Faculty Development Grant, \$750, \emph{Travel: Psychonomic Society Meeting in Chicago, Illinois}
\item 2015 (fall), Tarleton State University, OSRCA Undergrad. Res. Assistantship, \$1000, \emph{The effects of numerical fluency on mental representations of two-digit numbers.}
\item 2015 (fall), Tarleton State University, OSRCA Undergrad. Res. Assistantship, \$1000, \emph{Spatial-numerical associations in mental arithmetic.}
\item 2015 (summer), Tarleton State University, First Year Research Experience (FYRE), \$6500, \emph{Mental representations of two-digit numbers}
\item 2015 (summer), Tarleton State University, OSRCA Undergrad. Res. Assistantship, \$3500, \emph{Spatial-numerical associations in mental arithmetic.}
\item 2015 (spring), Tarleton State University, OSRCA Undergrad. Res. Assistantship, \$1000, \emph{Are the stages of cognitive arithmetic additive or interactive? The effects of numerical surface form on an addition production task.}
\item 2015 (spring), Society for the Teaching of Psychology, Early Career Travel Grant, \$350, \emph{Travel: Southwestern Teachers of Psychology Conference in Wichita, Kansas}
\item 2014-2015, Tarleton State University, Organized Research Grant, \$9560, \emph{Investigating the cognitive factors behind mathematics learning disability}
\item 2014 (fall), Tarleton State University, OSRCA Undergrad. Res. Assistantship, \$1000, \emph{The effects of numerical surface form on strategies for mental arithmetic verification}
\item 2014 (fall), Tarleton State University, Faculty Development Grant, \$508, \emph{Travel: Psychonomic Society Meeting in Long Beach, California}
\item 2014 (summer), Tarleton State University, OSRCA Undergrad. Res. Assistantship, \$3500, \emph{Using hand tracking to analyze mental representations of fractions}
\item 2014 (spring), Tarleton State University, QEP Startup Grant, \$1500, \emph{Applied Learning Experience: Undergraduate Research in Mathematical Cognition}
\item 2013 (fall), Tarleton State University, Faculty Development Grant, \$630, \emph{Travel: Psychonomic Society Meeting in Toronto, Ontario}
\item 2012-2013, National Science Foundation: Robert Noyce Scholarship Program, \$174,020 (Co-PI with Ben Jang), \emph{Building the Capacity for Math and Science Teacher Training}
\item 2010, Texas A\&M University – Commerce, OSP Research Grant, \$5000, \emph{Mouse Tracking in Mathematical Cognition}
\item 2008, Texas A\&M University – Commerce, OSP Mini Grant, \$600, \emph{IoLab Button Box for Psyscope X}
\end{itemize}

\subsection*{Editorial roles}
\label{sec-1-10}
\begin{itemize}
\item Associate Editor (2017-present): \emph{Journal of European Psychology Students}
\item Associate Editor (2017-present): \emph{Frontiers in Psychology: Cognition Section}
\item Associate Editor (2016-present): \emph{Journal of Psychological Inquiry}
\item Guest Editor (2016-present): \emph{Journal of Numerical Cognition}
\item Review Editor (2016-2017): \emph{Frontiers in Psychology: Cognition Section}
\end{itemize}
\subsection*{Reviewing}
\label{sec-1-11}

\begin{itemize}
\item Ad hoc reviewer for the following journals: \emph{Acta Psychologica, Attention, Perception, \& Psychophysics, Behavior Research Methods, British Journal of Developmental Psychology, Canadian Journal of Experimental Psychology, Cognitive Processing, Cognitive Science, Frontiers in Psychology, Journal of Experimental Child Psychology, Journal of Numerical Cognition, Learning and Individual Differences, Mathematics Teacher, Mathematics Teaching in the Middle School, Proceedings of the Research Council on Mathematics Learning, Psychonomic Bulletin and Review, Quarterly Journal of Experimental Psychology}
\item External Examiner
\begin{itemize}
\item 2017, Corinna Jones, Ph.D., University of Huddersfield, UK
\end{itemize}
\item Panelist/Reader
\begin{itemize}
\item 2017, Judge, American Statistical Association Statistics Project Competition
\item 2015-2017, Reader, AP Statistics Exam, Kansas City, MO
\item 2012-2015, Panelist, National Science Foundation, Washington, DC
\end{itemize}

\item Textbook reviewer for Psychology Press, Routledge, Sage, Taylor \& Francis
\end{itemize}

\subsection*{Professional Memberships}
\label{sec-1-12}
\begin{itemize}
\item American Mathematical Society
\item American Statistical Association
\item Canadian Society for Brain, Behaviour, \& Cognitive Science
\item Mathematical Cognition and Learning Society
\item Psychonomic Society
\item Society for Mathematical Psychology
\item Southwestern Psychological Association (SWPA)
\end{itemize}

\subsection*{Professional Service}
\label{sec-1-13}

National/regional service 

\begin{itemize}
\item Treasurer, Southwestern Psychological Association (appointed, 2017-2020)
\item Texas Representative, Southwestern Psychological Association (elected, 2015-2017)
\item Steering Committee Member, Southwestern Teachers of Psychology (appointed, 2015-2016)
\item Nominating Committee Chair, Southwestern Psychological Association (appointed, 2015)
\item Advisory Board Member, Collaborative Replications and Education Project (CREP) (appointed, 2014-2016)
\item Session Chair, Psychonomics Annual Meeting (2014, 2015)
\item Session Chair, Southwestern Psychological Association Meeting (2015)
\item Conference Committee Member, Research Council on Mathematics Learning (elected, 2012-2015)
\end{itemize}

University service

\begin{itemize}
\item Faculty Fellow, Tarleton (2016-present)
\item University ALE Task Force, Member, Tarleton (2016-present)
\item Official Advisor for Alpha Chi Honor Society, Tarleton (2015-present)
\item State Non-Funded Course Review Group, Member, Tarleton (2015-present)
\item Honors Advisory Committee for College of Education, Member, Tarleton (2015-present)
\item University Research Committee, Member, Tarleton (2015-present)
\item Curriculum Committee, College of Education, Member, Tarleton (2014-present)
\item Student Research and Creative Activity Advisory Committee, Member, Tarleton (2013-present)
\item Greater Expectations Task Force, Member, Tarleton (2014-2015)
\item Student Marshall, College of Education Commencement, Tarleton (Sp14, Sp15, Fa15)
\item Session Chair, Tarleton Research Symposium (2014, 2015)
\item Judge, Graduate Student Talks, Tarleton Research Symposium (2014, 2015)
\item External Search Committee member, Department of Engineering, TAMU-C (2012-2013)
\item University Honors Council, Member, TAMU-C (2012-2013)
\item Liberal Studies Committee, Member, TAMU-C (2012-2013)
\item Developmental Appeals Committee, Member, TAMU-C (2010-2012)
\end{itemize}

Department service

\begin{itemize}
\item Search Committee Chair, Department of Psychological Sciences, Tarleton (2014, 2015)
\item Texan Orientation, Department of Psychological Sciences, Tarleton (2014-2016)
\item Texan Tour Speaker, Department of Psychological Sciences, Tarleton (2015)
\item Organizer, Psychology Department Seminar, Tarleton (2013-present)
\item Psychology Scholarship Committee, Tarleton (2013-present)
\item Psychology Undergraduate Programs Committee, Member, TAMU-C (2011-2012)
\end{itemize}

\subsection*{Courses Taught}
\label{sec-1-14}

Tarleton State University

\begin{itemize}
\item PSYC 2301: General Psychology (Honors), Fall 13,14,15
\item PSYC 2301: General Psychology, Summer 14,15
\item PSYC 3301: Psychology of Learning, Fall 13,14,15,16
\item PSYC 3303: Educational Psychology, Fall 13; Spring 14; Summer 14,15,16
\item PSYC 3309: Writing in Psychology, Spring 16
\item PSYC 3320: Psycholinguistics, Summer 17
\item PSYC 3330: Elem Statistics for Behav Science, Fall 14,15,16; Spring 15,16,17; Su 16
\item PSYC 3435: Prin Research for Behav Science, Fall 14,15,16; Sp 15,16,17; Su 15,16
\item PSYC 4386: Advanced Statistical Methods, Spring 14
\item PSYC 4386: Methods in Experimental Psychology, Spring 15; Fall 15
\item PSYC 4386: Problems in Numerical Cognition, Fall 15
\item PSYC 5303: Theories of Learning, Fall 16
\item HONS 3385: Honors Seminar (Numerical Cognition), Spring 15
\item PSYC 5301: Research Methods, Spring 14,15,17
\item PSYC 5304: Human Development, Spring 14
\item PSYC 5379: Advanced Psycholinguistics, Summer 17
\item EDAD 6313: Statistical Methods for Educational Leadership, Spring 16
\end{itemize}

\subsection*{Student Mentoring}
\label{sec-1-15}

Masters Thesis Chair
\begin{itemize}
\item Chelsea Bradley (Applied Psychology, Tarleton, in progress)
\item Savannah Hines (Applied Psychology, Tarleton, in progress)
\end{itemize}

Doctoral Committees

\begin{itemize}
\item Matthew Herzberg (Educational Leadership, Tarleton, in progress)
\item Jeni McNeely (Educational Leadership, Tarleton, graduated 2016)
\item Trina Geye (Psychology, TAMU-C, graduated 2016)
\item Stephen McDaniel (Psychology, TAMU-C, in progress)
\item Beth Nikopoulous (Psychology, TAMU-C, graduated 2015)
\item Donna Peters (Psychology, TAMU-C, graduated 2013)
\end{itemize}

Masters Committees

\begin{itemize}
\item Trina Geye (Psychology, TAMU-C, graduated 2015)
\item Beth Nikopoulous (Psychology, TAMU-C, graduated 2013)
\item Heather Oetker (Special Education, TAMU-C, graduated 2012)
\item Joshua Patterson (Mathematics, TAMU-C, graduated 2011)
\end{itemize}

Honors Thesis Chair

\begin{itemize}
\item Paige Woodard (Psychology, Tarleton, graduated 2017)
\item Sarah Montgomery (High Honors in Psychology, TAMU-C, graduated 2013), \emph{Measuring the Working Memory Requirements of Mental Arithmetic}
\item Emily Dalton (Honors in Psychology, TAMU-C, graduated 2013), \emph{The Effects of Generation on False Memory for Numbers}
\item Kaytlin Reid (Honors in Interdisciplinary Studies, TAMU-C, graduated 2013), \emph{The Role of Working Memory in Mental Fraction Computation}
\item Douglas Boney (Honors in Mathematics, TAMU-C, graduated 2013), \emph{Knot Polynomials}
\item Samantha Reece (Honors in Sociology, TAMU-C, graduated 2012), \emph{The Effects of Stereotype Threat on Cheating Behavior in Mathematics}
\end{itemize}

Honors Thesis Committees

\begin{itemize}
\item Carmen Phelps (English, TAMU-C, 2013)
\item Morgan Lutz (Psychology, TAMU-C, 2013)
\item Nick Bredberg (Physics, TAMU-C, 2012)
\item Kallie Hinton (Mathematics Education, TAMU-C, 2011)
\item Lindsey Preston (Mathematics Education, TAMU-C, 2011)
\end{itemize}
% Emacs 25.2.1 (Org mode 8.2.10)
\end{document}